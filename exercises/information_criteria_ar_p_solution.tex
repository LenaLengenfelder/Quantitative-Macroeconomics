\begin{enumerate}
\item The first term measures the fit of a model with order \(p\) and it is the same for all criteria.
This term decreases for increasing order because there is no correction for degrees of freedom in the ML variance estimator.
	
The second term penalizes large AR orders.
How this term is effectively chosen distinguishes the different criteria.

The criteria have the following properties:
\begin{itemize}
    \item AIC asymptotically overestimates the order with positive probability
    \item HQC estimates the order consistently (\(plim~ \hat{p}= p\))
    \item SIC is even strongly consistent (\(\hat{p} \overset{a.s.}{\rightarrow} p\)) under quite general conditions
    (e.g.\ if the actual DGP is a finite-order AR process and the maximum order is larger than the true order).
\end{itemize}
In practice, one often relies on the AIC as having too many lags is not as severe as having too few lags.
This is especially true in small samples.

It is important to notice, that for correctly computing the information criteria
  one needs to use the same sample size across models with different orders.
In other words, the number of pre-sample values set aside for estimation is determined by the maximum order \(p^{\text{max}}\).

	
\item \lstinputlisting[style=Matlab-editor,basicstyle=\mlttfamily,title=\lstname]{progs/matlab/lagOrderSelectionARp.m}

\item[3./4.] \lstinputlisting[style=Matlab-editor,basicstyle=\mlttfamily,title=\lstname]{progs/matlab/AR4LagSelection.m}
\end{enumerate}