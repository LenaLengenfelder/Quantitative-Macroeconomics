\section[Some Fundamental Concepts Of Univariate Time Series Analysis]{Some Fundamental Concepts Of Univariate Time Series Analysis\label{ex:FundamentalConceptsUnivariateTimeSeriesAnalysis}}

\begin{enumerate}
\item Define the \enquote{White Noise Process} labeled shortly as \(\varepsilon_t \sim WN(0,\sigma_\varepsilon^2)\).
\item Plot 200 observations of 
\begin{align*} 
(i)~y_t  &= \varepsilon_t
\\
(ii)~y_t &= \frac{1}{5}(\varepsilon_{t-2}+\varepsilon_{t-1}+\varepsilon_{t}+\varepsilon_{t+1}+\varepsilon_{t+2})
\end{align*} 
with \(\varepsilon_{t} \sim N(0,1)\). What are the differences?	
\item Briefly explain the concepts of (i) weak stationarity and (ii) strict stationarity.
\item Define the autocovariance and autocorrelation function for a covariance-stationary stochastic process \( \{Y_t\} \).
\item Consider the linear first-order difference equation
\[y_t=\phi y_{t-1}+\varepsilon_t\]
  with \(\varepsilon_{t} \sim N(0,1)\).
Simulate and plot 200 observations of \((i) |\phi|<1\), \((ii) \phi=1\), and \((iii) |\phi| >1\).
What does this imply in terms of stationarity of the process?
\item Briefly explain the Lag-operator and Lag-polynomials.
How can we check whether an {AR{(p)}} process
\[y_t - \phi_1 y_{t-1} -\phi_2 y_{t-2} - \cdots  - \phi_p y_{t-p} = (1-\phi_1 L-\phi_2 L^2 -\cdots  - \phi_p L^p)y_t = \varepsilon_t\]
is weakly stationary?
\end{enumerate}

\paragraph{Readings}
\begin{itemize}
\item \textcite[Ch.2]{Bjornland.Thorsrud_2015_AppliedTimeSeries}
\item \textcite{Lutkepohl_2004_UnivariateTimeSeries}
\end{itemize}

\begin{solution}\textbf{Solution to \nameref{ex:FundamentalConceptsUnivariateTimeSeriesAnalysis}}
\ifDisplaySolutions%
\begin{enumerate}
\item A white noise has mean zero, a constant variance and all other second-order moments (i.e.\ autocovariances/autocorrelations) are zero:
\begin{align*}
    E[\varepsilon_t]&=0\\
    Var[\varepsilon_t]&=E[\varepsilon_t^2] - E[\varepsilon_t]E[\varepsilon_t] = \sigma_\varepsilon^2\\
    Cov(\varepsilon_{t},\varepsilon_s) &= E[\varepsilon_t \varepsilon_s] - E[\varepsilon_t]E[\varepsilon_s] = 0~\text{for}~s \neq t
\end{align*}	
\item \lstinputlisting[style=Matlab-editor,basicstyle=\mlttfamily,title=\lstname]{progs/matlab/whiteNoisePlots.m}
Every simulation is different, model can thus generate an infinite set of realizations over the period \(t=1,\ldots ,200\).
The processes do differ in their persistence.
(i) is the white noise process, which is not persistent.
(ii) is a 5-point-moving-average, which is a linear combination of white noise processes.
It is smoother and more persistent and very different from just noise.
Linear combinations of white noise processes build the basis of many models in time series analysis.
\item A process is said to be \textbf{\(N\)-order weakly stationary} if all its joint moments up to order \(N\) exist and are time invariant.
We are particularly interested in \(N=2\), i.e.\ \textbf{covariance stationarity}:
\begin{align*}
E[Y_t]&=\mu~\text{(constant for all t)}
\\
Var[Y_t]&=E[(Y_t - \mu)(Y_t-\mu)]=\gamma_0~\text{(constant for all t)}
\\
Cov[Y_{t_1},Y_{t_1-k}] &= E[(Y_{t_1}-\mu)(Y_{t_1-k}-\mu)] = Cov[Y_{t_2},Y_{t_2-k}] = \gamma_k~\text{(only dependent on \(k\))}
\end{align*}
That is the first two moments are not dependent on \(t\).
Particularly, the autocovariance is only dependent on the time difference \(k\), but not on the actual point in time \(t\).
\\
\textbf{Strict stationarity}: for all \(k\) and \(h\):
\[f(Y_t,Y_{t-1},\ldots ,Y_{t-k})=f(Y_{t-h},Y_{t-h-1},\ldots ,Y_{t-h-k})\]
That is, not only the first two moments but the whole distribution is not dependent on the point in time \(t\),
  but on the time difference \(k\).
\item Autocovariance function for a covariance-stationary process:
\[\gamma_k = E[(Y_t - \mu)(Y_{t-k}-\mu)]\]
where \(\gamma_0\) is the variance. Autocorrelation function: \[\rho_k = \gamma_k/\gamma_0\]
\\
We can estimate this by using:
\begin{align*}
\hat{\gamma}_k = c_k &= \frac{1}{T} \sum_{t=k+1}^T(y_t -\bar{y})(y_{t-k}-\bar{y})
\\
\hat{\rho}_k = r_k & = c_k/c_0
\end{align*}
Note: In most applications we don't correct the degrees of freedom for numerical reasons
  (e.g.\ to avoid singularity of autocovariance matrices in the multivariate case),
  i.e.\ the sums are not divided by \(T-k-1\) but simply by \(T\).
For \(T>100\) this does not really matter as the expressions are very close to each other.

\item \lstinputlisting[style=Matlab-editor,basicstyle=\mlttfamily,title=\lstname]{progs/matlab/plotsAR1.m}
Remarks: If \(|\phi|<1\) the series returns to the mean, i.e.\ it is stable and stationary.
If \(|\phi>1|\) then it explodes, i.e.\ it is unstable and not stationary.
\(\phi=1\) is a so-called random walk,
  it is the key model when working with non-stationary models.
Note that the random walk incorporates many different shapes, in macroeconomic forecasts we often want to \textbf{beat} the random walk model.
\item It is a special LINEAR operator, similar to the expectation operator, and very useful when working with time series.
The operator transforms one time series into another by shifting the observation from period \(t\) to period \(t-1\):
\(Ly_t = y_{t-1}\) or \(L^{-1} y_t =y_{t+1}\).
More general: \(L^k y_t = L^{k-1} L y_t = L^{k-1} y_{t-1} = \cdots  = y_{t-k}\).
Convenient use:
\[(1-L)y_t = y_t - y_{t-1}= \Delta y_t\]
We can also work with lag-polynomials:
\[ \phi(L) = (1-\phi_1 L-\phi_2 L^2 -\cdots  - \phi_p L^p)\]
where we call p the lag order. So:
\[ \phi(L) y_t = (1-\phi_1 L-\phi_2 L^2 -\cdots  - \phi_p L^p)y_t = y_t - \phi_1 y_{t-1} -\phi_2 y_{t-2} - \cdots  - \phi_p y_{t-p}\]
To check whether an {AR{(p)}} model is covariance stationarity, we need to check whether the roots of the lag-polynomial lie outside the unit circle.
That is, we treat \(L\) as a complex number \(z\in \mathbb{C}\) and compute the roots of \((1-\phi_1 z-\phi_2 z^2 -\cdots  - \phi_p z^p)=0\) (using a computer in most cases).
\end{enumerate}
\fi
\newpage
\end{solution}