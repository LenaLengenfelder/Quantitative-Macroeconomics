\section[Identification Problem in Structural Vector Autoregressive Models]{Identification Problem in Structural Vector Autoregressive Models\label{ex:IdentificationProblemSVAR}}
Consider a simple 2-variable model:
\begin{align*}
i_t &=  \beta \pi_t + \gamma_1 i_{t-1} + \gamma_2 \pi_{t-1} + \varepsilon_t^{MP} \\
\pi_t &=  \delta i_t + \gamma_3 i_{t-1} + \gamma_4 \pi_{t-1} + \varepsilon_t^{\pi}
\end{align*}
where \(i_t\) denotes the interest rate set by the central bank and \(\pi_t\) the inflation rate.
Assume for the \textbf{structural shocks}:
\(\varepsilon_{t}=(\varepsilon_t^{MP}, \varepsilon_t^{\pi})' \sim N(0,\Sigma_\varepsilon)\).
\begin{enumerate}
\item Rewrite the model in a compact matrix form \(B_0 y_t = B_1 y_{t-1} + \varepsilon_{t}\).
Note that this is a structural VAR(1) model.
\item Since the structural VAR model is not directly observable, derive the reduced-form representation:
\(y_t = A_1 y_{t-1} + u_t\).
What is the relationship between structural shocks \(\varepsilon_t\) and reduced-form residuals \(u_t\)?
\item In your own words, explain the identification problem in SVAR models.
Provide intuition behind the popular identification assumptions of short-run, long-run and sign restrictions.
\end{enumerate}

\paragraph{Readings}
\begin{itemize}
\item \textcite[Ch.~7.6]{Kilian.Lutkepohl_2017_StructuralVectorAutoregressive}
\end{itemize}

\begin{solution}\textbf{Solution to \nameref{ex:IdentificationProblemSVAR}}
\ifDisplaySolutions
\begin{enumerate}

\item Rewrite the equations:
\begin{align*}
i_t - \beta \pi_t  &=  \gamma_1 i_{t-1} + \gamma_2 \pi_{t-1} + \varepsilon_t^{MP}\\
\pi_t - \delta i_t &=  \gamma_3 i_{t-1} + \gamma_4 \pi_{t-1} + \varepsilon_t^{\pi}
\end{align*}
or in matrix notation:
\begin{align*}
\underbrace{\begin{pmatrix}	1 & -\beta \\ -\delta & 1 \end{pmatrix}}_{B_0} \underbrace{\begin{pmatrix} i_t \\ \pi_t \end{pmatrix}}_{y_t} = \underbrace{\begin{pmatrix} \gamma_1 & \gamma_2 \\ \gamma_3 & \gamma_4 \end{pmatrix}}_{B_1} \underbrace{\begin{pmatrix} i_{t-1} \\ \pi_{t-1} \end{pmatrix}}_{y_{t-1}} + \underbrace{\begin{pmatrix} \varepsilon_t^{MP} \\ \varepsilon_t^{\pi} \end{pmatrix}}_{\varepsilon_t}
\end{align*}

\item Pre-multiply both sides by \(B_0^{-1}\):
\begin{align*}
y_t = \underbrace{B_0^{-1} B_1}_{A_1} y_{t-1} + \underbrace{B_0^{-1} \varepsilon_t}_{u_t}
\end{align*}
Note that the reduced-form innovations \(u_t\) are a composite of the underlying structural shocks \(\varepsilon_t\):
\begin{align*}
u_t = B_0^{-1} \varepsilon_t
\end{align*}
The covariance matrices are related by:
\begin{align*}
E[u_t u_t'] = \Sigma_u = B_0^{-1} \Sigma_\varepsilon B_0^{-1'} = B_0^{-1} B_0^{-1'}
\end{align*}
Above, we make use of a normalization rule for \(\Sigma_\varepsilon=I\).
For the example above:
\begin{align*}
B_0 = \begin{pmatrix} 1 & -\beta \\ -\delta & 1 \end{pmatrix}
\\
B_0^{-1} = \frac{1}{\det(B_0)} \begin{pmatrix} 1 & \beta \\ \delta & 1 \end{pmatrix} \equiv \begin{pmatrix} a & b \\ c & d \end{pmatrix}
\end{align*}
So the system of equations that relates reduced-form innovations to structural shocks is given by:
\begin{align*}
u_t^{i} = a \varepsilon_t^{MP} + b \varepsilon_t^{\pi}
\\
u_t^{\pi} = c \varepsilon_t^{MP} + d \varepsilon_t^{\pi}
\end{align*}
Each reduced-form shock is a \textbf{weighted average} of structural shocks,
where \(a,b,c,d\) represents the amounts by which a particular structural shock contributes to the variation in each residual.

\item There is not enough information to solve this system of equations, because in \(B_0\) we have 4 unknowns,
but due to symmetry from \(\Sigma_u = B_0^{-1} B_0^{-1'}\) we only have 3 elements in \(\Sigma_u\): two variances and one covariance.
More generally, the covariance structure leaves \(K(K-1)/2\) degrees of freedom in specifying \(B_0^{-1}\)
and hence further restrictions are needed to achieve identification.

Some popular strategies:
\begin{enumerate}
\item Recursive ordering of variables (aka orthogonalization):
In the above example, we would set \(b=0\) to get a lower triangular \(B_0^{-1}\).
The \emph{economics} behind this choice is based on \emph{delay} assumptions,
i.e.\ how long it takes for a variable to react to a certain shock.
We can think of the structural shock in terms of the effect it exerts \textbf{contemporaneously} on the variable of interest:
\(\partial y_t = u_t = B_0^{-1} \varepsilon_t\), so we could write:
\begin{align*}
\begin{pmatrix} i_t \\ \pi_t \end{pmatrix} = \begin{pmatrix} a & 0 \\ c & d \end{pmatrix} \begin{pmatrix} \varepsilon_t^{MP} \\ \varepsilon_t^{\pi} \end{pmatrix}
\end{align*}
This lower triangular structure can be obtained by e.g.\ a Cholesky decomposition of \(\Sigma_u\) and yields \textbf{exact identification}.
The order of variables, however, matters!
\item Short-run restrictions: Exclusion restrictions on the impact matrix \(B_0^{-1}\), more flexible than orthogonalization.
\item Separating transitory from permanent components by assuming long-run structural relationships,
i.e.\ on the long-run multiplier matrix \({(I-A(L))}^{-1} B_0^{-1}\).
\item Combination of short-run and long-run relationships.
\item Sign restrictions: Take the Cholesky decomposition which yields exact identification \(\Sigma_u = B_0^{-1}B_0^{-1'} = P P'\).
In this special case: \({B_0}^{-1}=P\), but this is just \textbf{ONE} possible solution.
It is also possible to decompose \(\Sigma_u = \tilde{P}\tilde{P}'\), where \(\tilde{P} = PQ'\) and \(Q\) is an orthogonal rotation matrix: \(Q'Q = QQ'=I\);
that is, \(\tilde{P}\) and \(P\) are \textbf{observationally equivalent}, because they both reproduce \(\Sigma_u\).
\(Q\) is called a rotation matrix because it allows us to \emph{rotate} the initial Cholesky (recursive) matrix while maintaining the property that shocks are uncorrelated.
Put differently, it helps us generate new weights!
This is the basic idea of sign restrictions:
Examine a large number of candidate impact matrices by repeatedly drawing at random from the set of orthogonal matrices \(Q\).
For each \(B_0^{-1}\) check whether the candidate impact matrix is compatible with the sign restrictions that characterize a certain structural shock.
Then we construct the set of admissable models based on accepted draws.
\end{enumerate}

\end{enumerate}
\fi
\newpage
\end{solution}