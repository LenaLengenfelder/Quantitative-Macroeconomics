\begin{enumerate}
\item Short-run restrictions are often quite restrictive as we require knowledge of how certain variables react instantaneously to certain shocks.
Economic theory does not really give us much guidance in the short-run,
  so we usually argue that certain variables are \emph{sluggish}
  or information of variables is only available with a time lag.
If we can agree on such behavior, we can be pretty confident about these restrictions.
But most of the times, economists do not agree about the behavior of the variables in the short-run or have competing models about it.
On the other hand, there is much more agreement what happens in the long-run.
For instance, there is a vast literature on the effect of monetary policy shocks on output and inflation in the short-run with disagreeing results.
On the other hand, most economists would agree that the effect of demand shocks
  such as monetary policy shocks have no effects on output and a positive effect on the price level.
This suggests an alternative approach, i.e\ use theoretically-inspired long-run restrictions to identify shocks and impulse responses
  not only for the long-run behavior but also for the short-run behavior of all variables.

\item We have already shown that in a VAR(1) model we get the following structural impulse response function for horizons \(h=0,1,2,\cdots \)
\begin{align*}
    \frac{\partial y_t}{\partial \varepsilon_t'} = B_0^{-1}
    ,\qquad
    \frac{\partial y_{t+1}}{\partial \varepsilon_t'} = A_1 B_0^{-1}
    ,\qquad
    \frac{\partial y_{t+2}}{\partial \varepsilon_t'} = A^2_1 B_0^{-1}
    ,\qquad
    \frac{\partial y_{t+h}}{\partial \varepsilon_t'} = A^h_1 B_0^{-1}
\end{align*}
The effects of \(\Delta gdp_t\) are given by the first row and the second row contains the effects on \(ur_t\).
So the effect of the structural shocks on the covariance stationary variables \(ur_t\) and \(\Delta gdp_t\) is given by
\begin{itemize}
    \item \(B_0^{-1}\) on impact
    \item \(A_1 B_0^{-1}\) after one period
    \item \(A_1^2 B_0^{-1}\) after two periods
    \item \(A_1^h B_0^{-1}\) after \(h\) periods
\end{itemize}
Now, because \(ur_t\) and \(\Delta gdp_t\) are covariance stationary,
  the Eigenvalues of \(A_1\) are inside the unit circle.
So, in the long-run, \(h\rightarrow \infty \), we have that \(A^h\rightarrow 0\).
That is, the effect of the structural shocks on the covariance-stationary variables \(ur_t\) and \(\Delta gdp_t\) vanishes over time
  and they return to their expected mean value.
On the other hand, given the IRFs of \(\frac{\partial \Delta gdp_{t+h}}{\partial \varepsilon_t'}\)
  we can derive the IRF of \(gdp_{t+h}\) by the cumulative sum
\begin{align*}
\frac{\partial gdp_{t+h}}{\partial \varepsilon_t'} = \frac{\partial \Delta gdp_t}{\partial \varepsilon_t'} + \frac{\partial \Delta gdp_{t+1}}{\partial \varepsilon_t'} + \cdots  + \frac{\partial \Delta gdp_{t+h}}{\partial \varepsilon_t'}
\end{align*}
That is, the effect on the level of GDP to an increase in the structural shocks is equal to the \textbf{first row} in
\begin{itemize}
    \item \((I)B_0^{-1}\) on impact
    \item \((I+A_1) B_0^{-1}\) after one period
    \item \((I+A_1+A_1^2) B_0^{-1}\) after two periods
    \item \((I+A_1+A_1^2+A_1^3) B_0^{-1}\) after three periods
    \item \((I+A_1+A_1^2+\cdots +A_1^h) B_0^{-1}\) after $h$ periods
\end{itemize}
In the long-run, \(h\rightarrow\infty \), the effect of the structural shocks on the level of \(gdp_t\) is given by the \textbf{first row} of
\begin{align*}
    (I+A_1+A_1^2+A_1^3+\cdots ) B_0^{-1} = {(I-A)}^{-1} B_0^{-1}
\end{align*}
Again, we make use of the fact that \(ur_t\) and \(\Delta gdp_t\) are covariance-stationary,
  which implies that the Eigenvalues of \(A_1\) are inside the unit circle and we can make use of the formula for the geometric sum.

More generally, for VAR(p) models with variables in first-differences, we get the long-run effect matrix for the corresponding level variables
\begin{align*}
    \Theta(1) = {A(1)}^{-1} B_0^{-1}
\end{align*}
where \(A(1)= (I - A_1 - \cdots  - A_p)\) is the lag polynomial evaluated at \(L=1\).

\item Re-consider the long-run multiplier matrix:
\begin{align*}
\Theta(1) = {A(1)}^{-1} B_0^{-1} \equiv \begin{pmatrix} a & \textcolor{red}{c}\\ b & d \end{pmatrix}
\end{align*}
where \(a\) and \(c\) are the long-run effects of \(\varepsilon_t^{AS}\) and \(\varepsilon_t^{AD}\) on the level of \(gdp_t\).
Economic theory tells us that in the long-run \(gdp_t\) is not affected by an aggregate demand shock and it returns to its initial value.
In other words, economic theory requires that \(c=0\):
\begin{align*}
\Theta(1) = {A(1)}^{-1} B_0^{-1} \equiv \begin{pmatrix} a & \textcolor{red}{0}\\ b & d \end{pmatrix}
\end{align*}
Note that we don't have anything to say about $a$ (the effect of an AS shock on the level of GDP) and leave it unrestricted.
Similarly, as \(ur_t\) is covariance-stationary and does not enter \(y_t\) as first difference,
  we can't really identify or economically interpret \(b\) and \(d\) and leave it unrestricted as well.

\item There are two ways to infer \(B_0^{-1}\) from the restricted \(\Theta(1) = {A(1)}^{-1} B_0^{-1} \equiv \begin{pmatrix} a & \textcolor{red}{0}\\ b & d \end{pmatrix}\) matrix:
\\
\textbf{1. Method: Cholesky decomposition}
\\
Note that \(\Theta(1)\) is lower triangular and looks like a Cholesky factor. Indeed:
\begin{align*}
    \Theta(1) \Theta(1)' = {A(1)}^{-1} B_0^{-1} B_0^{-1'} {A(1)}^{-1'} = {A(1)}^{-1} \Sigma_u {A(1)}^{-1'}
\end{align*}
This right-hand side can be readily computed as both \(A(1)\) as well as \(\Sigma_u\) are given by the reduced-form estimation.
That is computing the lower triangular Cholesky factor of \({A(1)}^{-1} \Sigma_u {A(1)}^{-1'}\) provides \(\Theta(1)\).
From \(\Theta(1)\) we can compute \(B_0^{-1}\):
\begin{align*}
B_0^{-1} = A(1)\Theta(1)
\end{align*}
Once we have \(B_0^{-1}\) we can proceed as usual and compute the impulse response function.

\textbf{2. Method: Numerical optimization}
\\
Use a numerical optimizer to find a \(B_0^{-1}\) matrix that fulfills the following restrictions:
\begin{itemize}
    \item covariance restrictions: \(vech(B_0^{-1} B_0^{-1'}) = vech(\Sigma_u)\)
    \item long-run restriction: \(\Theta(1)_{1,2} = 0\)
\end{itemize}
where \(\Theta(1) = {A(1)}^{-1} B_0^{-1}\).
Note that both \(\Sigma_u\) as well as \({A(1)}^{-1}\) are fixed parameters and given by the reduced-form estimation.

\item The code might look like this:
\lstinputlisting[style=Matlab-editor,basicstyle=\mlttfamily,title=\lstname]{progs/matlab/BlanchardQuahLR.m}
Here is the helper function to impose the restrictions:
\lstinputlisting[style=Matlab-editor,basicstyle=\mlttfamily,title=\lstname]{progs/matlab/BlanchardQuahLR_f.m}
The supply shock has on impact a positive effect on the log-level of GDP, it then drops for one quarter, but increases again afterwards.
The long-run effect of a supply shock on the log-level of GDP is positive.
On the other hand, the demand shock obviously has no long-run effect on the log-level of GDP (due to our identifying restriction),
  whereas in the short-run the effect is negative.
The effects on the unemployment rate are basically flipped with the exception that in the long-run the effects vanish as \(ur_t\) is covariance-stationary.
\end{enumerate}