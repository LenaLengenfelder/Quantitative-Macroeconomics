\section[Law Of Large Numbers]{Law Of Large Numbers\label{ex:LawOfLargeNumbers}}
Let $Y_{1},Y_{2},\ldots $ be an i.i.d. sequence of arbitrarily distributed random variables
  with finite variance $\sigma_Y ^{2}$ and expectation $\mu$.
Define the sequence of random variables
\begin{equation*}
\overline{Y}_{T}=\frac{1}{T}\sum_{t=1}^{T}Y_{t}
\end{equation*}
\begin{enumerate}
\item Briefly outline the intuition behind the \enquote{law of large numbers}.
What are the differences between \enquote{almost-sure convergence}
  and \enquote{convergence in probability}?
\item Write a program to illustrate the law of large numbers for uniformly distributed random variables
  (you may also try different distributions such as normal, gamma, geometric, student's t with finite or infinite variance).
To this end, do the following:
\begin{itemize}
	\item Set $T=10000$ and initialize the $T \times 1$ output vector $u$.
	\item Choose values for the parameters of the uniform distribution.
	Note that $E[u] = (a+b)/2$, where $a$ is the lower and $b$ the upper bound of the uniform distribution.
	\item For $t=1,...,T$ do the following computations:
	\begin{itemize}
		\item Draw $t$ random variables from the uniform distribution with lower bound $a$ and upper bound $b$.
		\item Compute and store the mean of the drawn values in your output vector at position $t$.
	\end{itemize}
	\item Plot your output vector and add a line to indicate the theoretical mean of the uniform distribution.
\end{itemize}
\item Now suppose that the sequence $Y_{1},Y_{2},\ldots $ is an $AR(1)$ process:
$$Y_{t} =\phi Y_{t-1} +\varepsilon _{t}$$
where $\varepsilon _{t}\sim iid(0,\sigma _{\varepsilon }^{2})$ is not necessarily normally distributed and $|\phi |<1$.
Illustrate numerically that the law of large numbers still holds despite the intertemporal dependence.
\end{enumerate}

\paragraph{Readings}
\begin{itemize}
	\item \textcite[App. C]{Lutkepohl_2005_NewIntroductionMultiple}
	\item \textcite[App. C]{Neusser_2016_TimeSeriesEconometrics}
	\item \textcite{Ploberger_2010_LawLargeNumbers}
	\item \textcite[Ch. 3]{White_2001_AsymptoticTheoryEconometricians}
\end{itemize}

\begin{solution}\textbf{Solution to \nameref{ex:LawOfLargeNumbers}}
\ifDisplaySolutions
\begin{enumerate}
\item In probability theory, the law of large numbers (LLN) is a theorem
  that describes the result of performing the same experiment a large number of times
  (repetitions, trials, experiments, iterations, sample size).
According to the LLN, the average of the results obtained from a large number of times
  will be close to the theoretical expected value,
  and will tend to become closer as more iterations are performed.
There are different laws of large numbers that differ in the underlying assumptions on the stochastic process.
These laws are the cornerstones of asymptotic theory in statistics and econometrics.

In this exercise, the LLN is about determining what happens to \(\overline{Y}_T\) as \(T\rightarrow\infty\)
  (note that \(\overline{Y}_T\) is a random variable).
The LLN states that this series converges to the (unknown) expected value \(E[Y_t] = \mu \).
More precisely, the strong LLN implies that at the limit, we can exactly determine \( \mu \),
  whereas the weak LLN implies that we can only approximately determine \( \mu \),
  even though we can make the approximation very close to the unknown number \( \mu \).

Econometrically speaking:
\begin{itemize}
    \item Strong LLN means \textbf{almost-sure convergence}:\\
    At some point adding more observation does not matter at all for the average,
      \(\overline{Y}_T\) will be exactly equal to the expected value \(\mu \).
    That is, the sequence \(\overline{Y}_{1},\overline{Y}_{2},\ldots \) of random variables
      converges \textbf{almost surely} to the variable \(\mu \), if
    \begin{align*}
    \Pr\left( \left \{ \lim_{T\rightarrow \infty }\overline{Y}_{T}=\mu\right \} \right) =1
    \end{align*}
      or simply:
    \begin{align*}
    \overline{Y}_{T}\overset{a.s.}{\rightarrow }\mu
    \end{align*}
    This definition of convergence is not very important in Quantitative Macroeconomics.
				
    \item Weak LLN means that the sample mean \(\overline{Y}_T\) converges in probability to the population mean \(\mu \).
    That is, the sequence \(\overline{Y}_{1},\overline{Y}_{2},\ldots \) of random variables
      converges \textbf{in probability} to the variable \(\mu \), if
    \begin{align*}
    \lim_{T\rightarrow \infty }\Pr\left( |Y_{T}-\mu|<\delta \right) =1
    \end{align*}
    As \(T\rightarrow \infty \), the probability is approaching 1 very closely,
      but typically it will not be exactly equal to 1.
    In other words, the probability that the average is \enquote{far away}
      from the expectation \(\mu \) is zero,
      where we measure closeness by an arbitrary small number \(\delta>0\).
    More compact notation:
    \begin{eqnarray*}
    \overline{Y}_{T} &\overset{p}{\rightarrow }&\mu \\
    \textsl{plim}~\overline{Y}_{T} &=&\mu
    \end{eqnarray*}
    This definition of convergence is very important in Quantitative Macroeconomics.

    In Quantitative Macroeconomics, we are mainly concerned with identically distributed processes
      that are either independent of each other (like the white noise process)
      or that are homogenously dependent (like the VAR(1) process).
    Given assumptions on existence and boundedness of the unconditional moments of these processes,
      the weak LLN typically applies.
\end{itemize}

\item Here is an extended illustration for several distributions: 
\lstinputlisting[style=Matlab-editor,basicstyle=\mlttfamily,title=\lstname]{progs/matlab/lawOfLargeNumbers.m}
Note that for the \(t\)-distribution with infinite variance the weak LLN actually does not apply.

\item Here is an extended illustration for different error term distributions: 
\lstinputlisting[style=Matlab-editor,basicstyle=\mlttfamily,title=\lstname]{progs/matlab/lawOfLargeNumbersAR1.m} 
Note that we need to make sure that \(E[\varepsilon_t]=0\) when we simulate data.
We see that the weak law of large numbers holds under weaker conditions than iid.
For instance, one can show that for the stationary AR(1),
  necessary and sufficient conditions are:
  \(Var[y_t]<\infty\) and \(|\gamma(k)| \rightarrow 0\) as \(k \rightarrow \infty \).
This does not hold for all considered distributions in the code.
\end{enumerate}
\fi
\newpage
\end{solution}