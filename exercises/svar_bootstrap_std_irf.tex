\section[Bootstrapping Standard Deviations of Structural IRFs]{Bootstrapping Standard Deviations of Structural IRFs\label{ex:BootstrapStandardDeviationStructuralIRFs}}
Consider an exactly-identified structural VAR model subject to short- and/or long-run restrictions,
  where the structural impulse response of variable $i$ to shock $j$ at horizon $h$ are simply denoted as $\theta \equiv \Theta_{ij,h}$.
As an exact closed-form solution for the asymptotic standard errors of $\theta$ are only available under restrictive assumptions,
  we will rely on a numerical approximation using a bootstrap approach.
\begin{enumerate}
\item Reconsider an exercise (of your choice) from the lecture on SVAR models identified with exclusion restrictions and re-estimate the structural impulse response function.

\item Compute $\widehat{std}(\hat{\theta}^\ast)$ via a bootstrap approximation by following these steps:
\begin{itemize}
    \item Write a function \texttt{BootstrapGDP(VAR,opt)} which implements a standard residual-based bootstrap approach
      using sampling with replacement techniques on the residuals.
    Furthermore, the initial values should be drawn randomly in blocks.
    Hint: Use the companion form to do the simulations.
    \item Set bootstrap repetitions $B$ equal to 1000 (or higher)
      and initialize a $K \times K \times H \times B$ array \texttt{THETAstar},
      where the first dimension corresponds to variable $i=1,..,K$,
      the second dimension to shock $j=1,...,K$,
      the third dimension to the horizon of the IRFs $h=0,..,H$
      and the fourth dimension to the bootstrap repetition $b=1.,...,B$.
    \item For $b=1,...,B$ do the following (you may also try \texttt{parfor} instead of \texttt{for}
      in order to make use of Matlab's parallel computing toolbox if installed):
    \begin{itemize}
        \item Compute a bootstrap GDP $y_t^{b}$ using the function \texttt{BootstrapGDP(VAR,opt)}.
        \item Estimate the reduced-form and structural impulse response function on this artificial dataset
          with the same methodology, settings and identification restrictions as in the estimation of the original dataset.
        \item Store the structural IRFs in \texttt{THETAstar} at position \texttt{(:,:,:,b)}.
    \end{itemize}
	\item Compute the standard deviation of the bootstrap structural IRFs using \texttt{std(THETAstar,0,4)}.
\end{itemize}

\item Plot approximate 68\% and 95\% confidence intervals for the structural impulse response functions according to the delta method:
\begin{align*}
    \hat{\theta} \pm z_{\gamma/2} \widehat{std}(\hat{\theta}^\ast)
\end{align*}
where $z_{\gamma/2}$ is the $\gamma/2$ quantile of the standard normal distribution.
\end{enumerate}

\paragraph{Readings}
\begin{itemize}
	\item \textcite[Ch.~12.1-12.5, 12.9]{Kilian.Lutkepohl_2017_StructuralVectorAutoregressive}
\end{itemize}

\begin{solution}\textbf{Solution to \nameref{ex:BootstrapStandardDeviationStructuralIRFs}}
\ifDisplaySolutions
\begin{enumerate}

\item[1.] We will re-consider the \textcite{Rubio-Ramirez.Waggoner.Zha_2010_StructuralVectorAutoregressions} example.

\item[2./3.]  The helper function to generate bootstrap DGPs might look like this:
\lstinputlisting[style=Matlab-editor,basicstyle=\mlttfamily,title=\lstname]{progs/matlab/bootstrapDGP.m}
The main script might look like this:
\lstinputlisting[style=Matlab-editor,basicstyle=\mlttfamily,title=\lstname]{progs/matlab/bootstrappingStdIRFs_RWZSRLR.m}
\end{enumerate}

\fi
\newpage
\end{solution}