\begin{enumerate}
\item \emph{git} is a version control system, i.e. a way to track changes to code, text, documents, data etc.
	It let's you go back and forth between many different versions of the same file, and see a list of the differences.
	Collaboration becomes (technically) very easy and straightforward as people can work on different files or different versions of the same file simultaneously
  	  and afterwards merge their changes.
	
	\emph{git} is the most popular version control system invented in 2005 to track the development of the worldwide largest open-source project: the Linux Kernel.
    It is a \textbf{command line tool}, and at some point you should learn the commands on the Command Line Interface (CLI).
	However, there are many graphical user interface (GUI) programs that make getting started with Git much easier
	  and integrate seamlessly with online collaboration platforms such as GitHub or GitLab.
	Therefore, in this exercise I will focus on such a tool called GitKraken,
	  which I use daily and highly recommend.\footnote{Other GUI programs work very similarly,
	  see \url{https://en.wikipedia.org/wiki/Comparison_of_Git_GUIs} for a comparison of features.
	  I also recommend the built-in git functionality of Visual Studio Code.}
	GitKraken offers a free trial of their paid license,
	  but also offers a free version for use on publicly-hosted repositories (which is fine for our purposes as we will mostly do stuff locally anyway).
	Additionally, GitKraken also offers the Pro license FREE to students and teachers through the GitHub Student Developer Pack (\url{https://education.github.com/pack}).
	
	While git and GitKraken are tools that you install on your computer,
      GitHub is an online platform that provides a nice visual interface to help you manage your version-controlled projects remotely.
    It is the largest git repository hosting service and has become by far the largest open-source collaboration site.	
	Another important online platform is Gitlab as you can also host that on your computer or server.
	For individuals gitea offers yet another way to host a stripped down version of GitHub or Gitlab.	
	I personally have accounts on GitHub (\url{https://github.com/wmutschl}) and Gitlab (\url{https://gitlab.com/wmutschl}),
	  but also use self-hosted versions of Gitlab (\url{https://git.dynare.org/wmutschl}) and gitea (\url{https://git.mutschler.eu}) to mirror my projects.
\item A typical workflow looks like this:
	\begin{itemize}
	\item retrieve data and prepare it for estimation purposes
	\item select a model framework, decide on certain hyperparameters and modeling choices and then run an estimation
	\item prepare tables, graphs and reports
	\end{itemize}
	All of these tasks heavily rely on coding, i.e. putting text into some files that are then evaluated by software
	  that actually performs the tasks.
	Moreover, we will see that estimating macroeconomic models requires a lot of trial and error and accordingly those files constantly change and need to be adapted.
	\emph{git} enables you to track these changes as it gives you an organized revision history.
	So you can experiment with your codes, make changes to a project and always keep the ability to go back and fourth between changes.
	So stop naming files like \emph{2022-10-17-master-thesis-v2-final-now-really-final.tex}
	  and let \emph{git} do its magic for you by simply tracking the file \emph{thesis.tex} with all of its revision history.
\item Follow the instructions provided in the links or get in touch if you are struggling with the installation.
\item Follow the instructions provided in the links or get in touch if you are struggling with the installation.
\item Follow the instructions provided in the links or get in touch if you are struggling with the installation.
\item In GitKraken: Open a new Tab, click \emph{Start a local repo}, then on the \emph{Init} register select \emph{Local Only} and fill out the details.
  Note that GitKraken automatically creates a first commit with a \texttt{README.md} file.
  Inside every repository there is a hidden folder \texttt{.git}.
  It contains everything done by \emph{git}, so all the changes you will ever do.
  Never delete this folder!
  Also putting a repository on a cloud storage folder might damage this folder,
    so best practice is to use a local folder on the disk.
  We will cover how to push the repository to a so-called remote which works basically like syncing,
	but much more robust and git-ier.
\item Now the benefits of using a GUI like GitKraken become evident,
  as our changes are displayed in the \emph{Unstaged Files} area
  and by clicking on the file we get a really pretty side-by-side comparison of all the changes.
  We can now decide which lines we want to \texttt{stage} and \texttt{commit}.
\item The git model looks like the following diagram:\\
\begin{center}

\begin{tikzpicture}[mypostaction/.style 2 args={
	decoration={
		 text align={
			   left indent=#1},
			   text along path, 
			   text={#2}
			   },
	  decorate
   }
]
	\node[draw,		
		fill=Rhodamine!50,
		minimum width=3.5cm, 
	 	minimum height=3cm,
		text width=3cm,
		text centered
	] (unstaged) at (0,0){File changes in working directory (Unstaged Files)};
	 
	\node [draw,
		fill=Goldenrod,
		minimum width=3.5cm,
		minimum height=3cm,
		text width=3cm,
		text centered,
		right=1cm of unstaged
	]  (staged) {Staged files};
	 
	\node [draw,
		fill=SpringGreen, 
	 	minimum width=3.5cm, 
	 	minimum height=3cm,
		text width=3cm,
		text centered,
	 	right=1cm of staged
	] (local) {Local repository};
	 
	\node [draw,
		fill=SeaGreen, 
		minimum width=3.5cm, 
		minimum height=3cm,
		text width=3cm,
		text centered,
		right=1cm of local
	]  (remote) {Remote repository (GitHub, GitLab) [optional]};
	 
	% Arrows with text label
	\coordinate (unstageRoot) at (-0.5,2); \coordinate (stageRoot) at (4,2);
	\draw[-latex, blue!20!white, line width=2ex]  (unstageRoot) to[in=135,out=90] (stageRoot);	
	\path [postaction={mypostaction={1cm}{stage changes}},postaction={mypostaction={1.5cm}{git add},/pgf/decoration/raise=-3mm}](unstageRoot) to [in=180,out=3] (stageRoot);

	\coordinate (stageRoot) at (4.5,2); \coordinate (localRoot) at (9,2);
	\draw[-latex, blue!20!white, line width=2ex]  (stageRoot) to[in=135,out=90] (localRoot);	
	\path [postaction={mypostaction={1cm}{commit changes}},postaction={mypostaction={1.5cm}{git commit},/pgf/decoration/raise=-3mm}](stageRoot) to [in=180,out=3] (localRoot);

	\coordinate (localRoot) at (9.5,2); \coordinate (remoteRoot) at (14,2);
	\draw[-latex, blue!20!white, line width=2ex]  (localRoot) to[in=135,out=90] (remoteRoot);	
	\path [postaction={mypostaction={1cm}{push changes}},postaction={mypostaction={1.5cm}{git push},/pgf/decoration/raise=-3mm}](localRoot) to [in=180,out=3] (remoteRoot);
	\end{tikzpicture}
\end{center}
	You do your work in your \textbf{working directory}.
	On the \textbf{stage} you collect all the changes that you want to save.
	This is very powerful because sometimes it is just individual lines of code or text that you want to keep track of and not the whole file.
	Once you've tracked all the changes that you want to combine, it is time to collect these changes into a \textbf{commit}.
	A commit is a permanent snapshot of the files that git tracks stored in the .git directory. It is associated with a unique identifier (hash).
	In other words, a commit is like a snapshot in time; you can always revert back to this and see what changes were made compared to any other commit.
	On your local repository (i.e. on your local machine) you now have a nice versioned history.
	However, if you want to collaborate with others or sync your repository to a specialized cloud provider you need to push these changes to a so-called remote repository,
	  typically on GitHub, GitLab, but any folder that you can access via remotely might serve as a remote repository.
\item Click on the file and select \texttt{Stage File} or add each line by clicking on the plus or minus signs left to each line.
Once you are happy with the file, click on the X to close the file-comparison window.
We now don't see any unstaged files and can proceed to write a commit message and then click on the big green button.
\item A \emph{good commit} typically does one discrete task or change only.
	For example, you added a variable to the regression specification in the code, in the output and in the report.
	Or you changed the name of a variable and treat it properly across multiple scripts.
	This enables you to make meaningful commit messages like \emph{Add year dummies to regression specification}
	and you thus end up with a well organized repository.
	This workflow needs some practice and everyone is slightly different with regards to this.
	Nevertheless, try to combine changes to certain meaningful smaller tasks and provide good commit messages.
	In my experience, having ten tiny commits is always preferable to one large commit.
	Your future self and collaborators will thank you!

	The question to what you should include in your commits,
	  is also a matter of choice and preference.
	Definitely your script files of codes, latex and text files.
	Data is also sometimes given as csv files which are basically just text files.
	Binary files (like Excel sheets, Word documents, Power Point slides) are a bit tricky to handle,
	  as you can't see the differences between versions in git.
	It depends on the specific needs whether one should commit these files as well (e.g. for Excel files with data this obviously makes sense),
	  but I usually don't do this.
	Note that GitHub doesn't allow files larger than 100 MB or projects with total size larger than 1 GB.
	There is also a way to deal with large binary files called \texttt{Git Large File Storage (LFS)}, but we won't need this.
\item Right click on the initial commit and select \emph{Reset main to this commit - Soft}.
Click on the file in the staged files section and remove the last line from the stage.
Re-commit your stage by providing a meaningful commit message and hitting the green button.
Click on Stash to put the remaining changes into the stash.
\item Simply click on Push and add the remote. On the left Panel click on REMOTE to see the current remote (usually named origin).
Note that you can add several remotes (say from different people) and compare the commits.
Remotes are also a nice backup of your codes.
\item Branches are arguably the most powerful part of \emph{git}.
By default you have a \textbf{main} branch,
  but what if you want to do some experiments, re-write an estimation function from scratch, work on a new feature, etc?
You could copy the whole folder and start working there or you use git and create a branch and make the changes there.
You can switch between branches, make commits to any branch, move them around, etc.
If your experiment doesn't work out, simply delete the branch.
If your experiments work out, commit them and merge them into the main branch.
Sometimes there will be conflicts which one needs to sort out,
  but using GUI tools like GitKraken makes this very easy
  as you have a pretty side-by-side comparison of changes.
Branches are arguably the most powerful part of \emph{git} especially for our purposes
  as research is a highly nonlinear process, and this way of doing version control is much more similar to how we actually work
  than the very linear way that other cloud storage providers do version control.
Branches are also extremely powerful for collaboration
  as different people can work on the same thing at the same time.

Select a so-called parent commit, where you want to create a new branch.
Note that this doesn't have to be the latest commit.
Click on the button \emph{Branch} and name it according to the exercise.
On the left panel, click on LOCAL to see an overview of all your branches.
\item Create, copy and paste the three files into your repository.
Check for pasting errors and then \emph{Stage all changes} and commit them.
\item Run the commands and solve any errors you might get from latex.
\item Follow the instructions in the exercise.
Note that there is a difference between \enquote{Ignore} and \enquote{Ignore and Stop Tracking}.
\enquote{Ignore} simply adds the file(type) to the \texttt{.gitignore} file so that new files with that name/type/whatever are not tracked.
To \enquote{Ignore and Stop Tracking} means to remove the file(s) from git version control:
  they will no longer be in the repo (as of the commit that performs the "stop tracking").
Basically, use \enquote{Ignore and Stop Tracking} if the file(s) you are ignoring never should have been in the repo in the first place.
\item Make sure you are on the correct branch \emph{latex-exam-template}
  and push this branch to GitHub.
Either right click on the commit or go to the left panel, click on PULL REQUESTS and on the green plus sign that appears.
Select the \emph{latex-exam-template} as the FROM REPO branch and \emph{main} as the TO REPO branch.
Enter a Title and Description and click on the green button.
Have a look in GitHub ar the pull request.
As there are no conflicts merge it and go back to GitKraken to see what happens in your repository.
You might need to \enquote{fetch origin} by right clicking on the origin remote.
\item Double click on your local main branch and then click on pull,
  which fast forwards your repo to the merged changes.
  Then click on Pop to get the WIP codes which were stored on main.
  Right click on the README.md file in the \emph{Unstaged Files} area and select \emph{Discard changes}.
\end{enumerate}