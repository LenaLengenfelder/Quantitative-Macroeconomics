\begin{enumerate}
\item Equation~\eqref{eq:NominalInterestRate} captures that bond prices are inversely related to interest rates.
When the interest rate goes up, the price of bonds falls.
Intuitively, this makes sense because if you are paying less for a fixed nominal return (at par),
  your expected return should be higher.
More specifically to our model, we consider so-called zero-coupon bonds or discount bonds.
These bonds don't pay any interest but derive their value from the difference between the purchase price and the par value (or the face value) paid at maturity.
On maturity the bondholder receives the face value of his investment.
So instead of interest payments, you get a large discount on the face value of the bond; that is the price is lower than the face value.
In other words, investors profit from the difference between the buying price and the face value, contrary to the usual interest income.
In our model, we consider zero-coupon bonds with a face value of 1.
So suppose that you buy such a bond at a price of 0.8, then although the bond pays no interest,
  your compensation is the difference between the initial price and the face value.
Let \(R_t\) denote the \emph{gross yield to maturity} of a zero-coupon bond,
  that is the discount rate that sets the present value of the promised bond payments equal to the current market price of the bond.
So the price of a Zero-Coupon bond is equal to
\begin{align*}
Q_t = \frac{1}{R_t}
\end{align*}
In our example this would imply \(R_t=1.25\).
As there are no other investment opportunities in this model \(R_t\) is also equal to the nominal interest rate in the economy.
	
Equation \eqref{eq:RealInterestRate} is the so-called Fisherian equation
  which states that the gross real return on a bond \(r_t\) is equivalent to the gross nominal interest rate divided by the expected gross inflation rate.
Inflation expectations are responsible for the difference between nominal and real interest rates, showing that future expectations matter for the economy.

\item In equilibrium, bond-holding is always zero in all periods: \(B_t=0\).
This is due to the fact that in this model we have a representative agent and only private bonds.
If all agents were borrowing, there would be nobody they could be borrowing from.
If all were lenders, nobody would like to borrow from them.
In sum the price of bonds (or more specifically the nominal interest rate) adjusts such
  that bonds across all agents are in zero net supply as markets need to clear in equilibrium.
Note, though, that this bond market clearing condition is imposed \emph{after} you derive the households optimality conditions
  as household savings behavior in equilibrium still needs to be consistent with the bond market clearing.

\item The \emph{No-Ponzi-Game} or \emph{solvency} condition is an external constraint imposed on the individual by the market or other participants.
You forbid your agent from acquiring infinite debt that is never repaid, a so-called Ponzi-scheme.
That is, the individual is restricted from financing consumption by raising debt and then raising debt again to repay the previous debt and finance again consumption and so on.
The individual would very much like to violate it, though, so we need to impose this constraint.
In short: the \emph{solvency} condition prevents that households consume more than they earn and refinance their additional consumption with excessive borrowing.

The \emph{transversality condition} is an optimality condition that states that it is not optimal to start accumulating assets and never consume them, i.e.
\(\lim_{T \rightarrow \infty} E_t \left \{ \Lambda_{t,T} \frac{B_T}{P_T} \right \} \leq 0\).
But with respect to optimality you would still want to run a Ponzi-scheme if allowed one.
\(\lim_{T \rightarrow \infty} E_t \left \{ \Lambda_{t,T} \frac{B_T}{P_T} \right \} \leq 0\)
combined with
\(\lim_{T \rightarrow \infty} E_t \left \{ \Lambda_{t,T} \frac{B_T}{P_T} \right \} \geq 0\)
yields \(\lim_{T \rightarrow \infty} E_t \left \{ \Lambda_{t,T} \frac{B_T}{P_T} \right \} = 0\).
This condition must be satisfied in order for the individual to maximize intertemporal utility implying that at the limit wealth should be zero.
In other words, if at the limit wealth is positive it means that the household could have increased its consumption without necessarily needing to work more hours;
  thus implying that consumption was not maximized and therefore contradicting the fact that the household behaves optimally.
In short: transversality conditions make sure that households do no have any leftover savings (in terms of bonds or capital)
  as this does not correspond to an optimal path of utility-enhancing consumption.
	
We never need to actually include this condition into our codes, but implicitly we use it to pick a certain steady state or trajectory.
For instance in the RBC model we have three possible steady states \((k_t=0,c_t=0)\), \(k_t>0,c_t=0\) or \(k_t>0,c_t>0\).
We do not consider the first one because in this case the economy does not exist.
All the trajectories leading to the second one violate the transversality condition,
  so finally we select the third steady state as the \emph{good one} and this is exactly the one that is most interesting from an economic point of view.

Coming back to our model, both the \emph{solvency} and \emph{transversality condition} are actually full-filled already
  as bond-holding is always zero in all periods including the hypothetical asymptotic end of life: \(B_t=0\) for all \(t\).
So these conditions are rather trivial in this model setting, but are important in more sophisticated models.

\item The household minimizes consumption expenditures \(\int_0^1 P_t(h) c_t(h)dh\) by choosing \(c_t(h)\) and taking the aggregation technology into account.
That is, the Lagrangian is given by:
\begin{align*}
	\pounds^c = \int_0^1 P_t(h) c_t(h)dh + P_t \left( c_t - {\left[\int_0^1 {(c_t(h))}^{\frac{\epsilon-1}{\epsilon}}dh\right]}^{\frac{\epsilon}{\epsilon-1}} \right)
\end{align*}
where \(P_t\) denotes the Lagrange multiplier, i.e.\ the cost of an additional unit in the index \(c_t\).
Setting the derivative with respect to \(c_t(h)\) equal to zero yields:
\begin{align*}
	\frac{\partial \pounds^c}{\partial c_t(h)} = P_t(h) - P_t \left(\frac{\epsilon}{\epsilon-1}\right) \underbrace{{\left[\int_0^1 {(c_t(h))}^{\frac{\epsilon-1}{\epsilon}}dh\right]}^{\frac{\epsilon}{\epsilon-1}-1}}_{c_t^{1/\epsilon}} \left(\frac{\epsilon-1}{\epsilon}\right) {(c_t(h))}^{\frac{\epsilon-1}{\epsilon}-1} = 0
\end{align*}
which can be simplified to:
\begin{align*}
	c_t(h) = {\left(\frac{P_t(h)}{P_t}\right)}^{-\epsilon} c_t
\end{align*}
Note that this is the demand function for each consumption good \(c_t(h)\).
Accordingly, \(\epsilon \) is the (constant) demand elasticity.
	
Plugging this expression into the aggregation technology yields:
\begin{align}
	&c_t^{\frac{\epsilon-1}{\epsilon}} = \int_0^1 {(c_t(h))}^{\frac{\epsilon-1}{\epsilon}}dh = \int_0^1 {\left({\left(\frac{P_t(h)}{P_t}\right)}^{-\epsilon} c_t\right)}^{\frac{\epsilon-1}{\epsilon}}dh = c_t^{\frac{\epsilon-1}{\epsilon}} P_t^{\epsilon-1} \int_0^1 {(P_t(h))}^{1-\epsilon}dh \nonumber
	\\
	\Leftrightarrow &
	P_t = {\left[\int_0^1 {(P_t(h))}^{1-\epsilon}dh\right]}^{\frac{1}{1-\epsilon}} \nonumber
	\\
	\Leftrightarrow &
	1 = \int_0^1 {\left(\frac{P_t(h)}{P_t}\right)}^{1-\epsilon}dh \label{eq:AggregatePriceIndex}
\end{align}
Similar to the aggregation technology for the consumption index \(c_t\),
  \(P_t\) can be interpreted as the aggregation technology for the different prices \(P_t(h)\).

In the budget constraint, we can now get rid of one integral \(\int_0^1 c_t(h) P_t(h) dh = P_t c_t\), because:
\begin{align*}
	\int_0^1 c_t(h) P_t(h) dh = \int_0^1 {\left(\frac{P_t(h)}{P_t}\right)}^{-\epsilon} c_t P_t(h) dh
	= P_t c_t \underbrace{\int_0^1 {\left(\frac{P_t(h)}{P_t}\right)}^{1-\epsilon}  dh}_{\overset{\eqref{eq:AggregatePriceIndex}}{=1}} = P_t c_t
\end{align*}
That is, conditional on optimal behavior of households,
  total consumption expenditures can be rewritten as the product of the aggregate price index times the aggregate consumption quantity index.

\item Due to our assumptions, the solvency and transversality conditions as well as the concave optimization problem,
  we can rule out corner solutions and neglect the non-negativity constraints in the variables and the budget constraint;
  hence, we only need to focus on the first-order conditions.
The Lagrangian for the household's problem is:
\begin{align*}
	\pounds^{HH} &= E_t \sum_{j=0}^{\infty} \beta^j \left \{	U\left({c}_{t+j},l_{t+j},z_{t+j}\right) \right \}
	\\
	& + \beta^j \lambda_{t+j} \left \{ \int_0^1 div_{t+j}(f) df + w_{t+j} l_{t+j} + \underbrace{\frac{B_{t-1+j}}{P_{t-1+j}}}_{b_{t-1+j}} \underbrace{\frac{P_{t-1+j}}{P_{t+j}}}_{\Pi_{t+j}^{-1}} - Q_{t+j} \underbrace{\frac{B_{t+j}}{P_{t+j}}}_{b_{t+j}} - c_{t+j} 
	\right \}
\end{align*}
where \(\beta^j\lambda_{t+j}\) are the Lagrange multipliers corresponding to period \(t+j\)'s \textbf{real} budget constraint
  (be aware of the difference between nominal and real variables and constraints; for instance, \(b_t=B_t/P_t\) is real debt).
The problem is not to choose \({\{c_t,l_t,b_{t}\}}_{t=0}^\infty \) all at once in an open-loop policy,
  but to choose these variables sequentially given the information at time \(t\) in a closed-loop policy,
  i.e.\ at period \(t\) decision rules for \({\{c_t,l_t,b_{t}\}}\) given the information set at period \(t\);
   at period \(t+1\) decision rules for \({\{c_{t+1},l_{t+1},b_{t+1}\}}\) given the information set at period \(t+1\), etc.
	
\paragraph{First-order condition with respect to \(c_t\)}
\begin{align}
	\lambda_t = \frac{\partial U(c_t,l_t,z_t)}{\partial c_t} = z_t c_t^{-\sigma}
\end{align}
This is the marginal consumption utility function, i.e.\ the benefit (shadow price) of an additional unit of revenue (e.g.\ dividends or labor income) in the budget constraint.
	
\paragraph{First-order condition with respect to \(l_t\)}
\begin{align}
	w_t = - \frac{\partial U(c_t,l_t,z_t)/\partial l_t}{\lambda_t} = - \frac{\partial U(c_t,l_t,z_t)/\partial l_t}{\partial U(c_t,l_t,z_t)/\partial c_t} = l_t^{\varphi} c_t^{\sigma}
\end{align}
This is the \textbf{intratemporal} optimality condition or, in other words, the labor supply curve of the household.
Note that the preference shifter \(z_t\) has no effect on this intratemporal decision.
	
\paragraph{First-order condition with respect to \(b_t\)}
\begin{align}
	\lambda_t Q_t &= \beta E_t \left[\lambda_{t+1} \Pi_{t+1}^{-1} \right]
\end{align}
Combined with \eqref{eq:NominalInterestRate} and \eqref{eq:RealInterestRate} this yields the so-called Euler equation,
  i.e.\ the \textbf{intratemporal} choice between consumption and saving:
\begin{align}
	\frac{\partial U(c_t,l_t,z_t)}{\partial c_t} &= \beta E_t \left[\frac{\partial U(c_{t+1},l_{t+1},z_{t+1})}{\partial c_{t+1}} R_t \Pi_{t+1}^{-1} \right]
	\\
	z_t c_t^{-\sigma} &= \beta E_t \left[z_{t+1} c_{t+1}^{-\sigma} \right] r_t
	\label{eq:HH.Euler}	
\end{align}
In words, intertemporal optimality is characterized by an indifference condition:
An additional unit of consumption yields either marginal utility today in the amount of \(\frac{\partial U(c_t,l_t,z_t)}{\partial c_t}\) (left-hand side).
Or, alternatively, this unit of consumption can be saved given the real interest rate \(r_t\).
This saved consumption unit has a present marginal utility value of \(\beta E_t \left[z_{t+1} c_{t+1}^{-\sigma} \right] r_t\) (right-hand side).
An optimal allocation equates these two choices.

\item The output packers maximize profits \(P_t y_t - \int_{0}^{1} P_t(f) y_t(f) df\) subject to~\eqref{eq:Firms_DS_Aggregator}.
The Lagrangian is
\begin{align*}
	\pounds^{p} = P_t y_t - \int_{0}^{1} P_t(f) y_t(f) df + \Lambda_t^p \left \{ {\left[\int\limits_0^1 {(y_t(f))}^{\frac{\epsilon-1}{\epsilon}}df\right]}^{\frac{\epsilon}{\epsilon-1}} - y_t \right \}
\end{align*}
where \(\Lambda_t^p\) is the Lagrange multiplier corresponding to the aggregation technology.
The first-order condition w.r.t \(y_t\) is
\begin{align}
	P_t = \Lambda_{t}^p
\end{align}
\(\Lambda_{t}^p\) is the gain of an additional output unit; hence, equal to the aggregate price index \(P_t\).

The first-order condition w.r.t \(y_t(f)\) yields:
\begin{align}
	\frac{\partial \pounds^{p} }{\partial y_t(f)} & = -P_t(f) + \Lambda_{t}^p \frac{\epsilon}{\epsilon-1} \left[\int_{0}^1 y_t(f)^{\frac{\epsilon-1}{\epsilon}}df\right]^{\frac{\epsilon}{\epsilon-1}-1} \frac{\epsilon-1}{\epsilon} {(y_t(f))}^{\frac{\epsilon-1}{\epsilon}-1} = 0
\label{eq:Firms_Relative_Demand}
\end{align}
Note that \(\left[\int_{0}^1 {(y_t(f))}^{\frac{\epsilon-1}{\epsilon}}df\right] = y_t^{\frac{\epsilon-1}{\epsilon}}\) and \(\Lambda_{t}^p=P_t\).
Therefore:
\begin{align}
	P_t(f) = P_t {\left[y_t^{\frac{\epsilon-1}{\epsilon}}\right]}^{\frac{\epsilon-\epsilon+1}{\epsilon-1}} {(y_t(f))}^{\frac{\epsilon-1-\epsilon}{\epsilon}} = P_t {\left(\frac{y_t(f)}{y_t}\right)}^{\frac{-1}{\epsilon}}
\end{align}
Reordering yields
\begin{align}
	y_t(f) = {\left(\frac{P_t(f)}{P_t}\right)}^{-\epsilon} y_t \label{eq:firms_demand}
\end{align}
This is the demand curve for intermediate good \(y_t(f)\).
Again we see that \(\epsilon \) is the constant demand elasticity.
	
The aggregate price index is implicitly determined by inserting the demand curve~\eqref{eq:firms_demand} into the aggregator~\eqref{eq:Firms_DS_Aggregator}
\begin{align}
	y_t &= {\left[\int\limits_0^1 {\left({\left(\frac{P_t(f)}{P_t}\right)}^{-\epsilon} y_t\right)}^{\frac{\epsilon-1}{\epsilon}}df\right]}^{\frac{\epsilon}{\epsilon-1}}
\\
\Leftrightarrow
	P_t &= {\left[\int_{0}^{1} {(P_t(f))}^{1-\epsilon}df\right]}^{\frac{1}{1-\epsilon}} \label{eq:Firms_DS_Pricing}
\end{align}

\item As the firms are owned by the households, the nominal stochastic discount factor,
  \(\Lambda_{t,t+j}\), between \(t\) and \(t+j\) is derived from the Euler equation~\eqref{eq:HH.Euler} of the households
  \(\lambda_t = \beta E_t \left[\lambda_{t+1} R_t \Pi_{t+1}^{-1} \right]\)
  which implies for the stochastic discount factor:
\begin{align*}
	E_t \Lambda_{t,t+j} = E_t 1/R_{t+j} = E_t\beta^j \frac{\lambda_{t+j}}{\lambda_{t}}\frac{P_t}{P_{t+j}}
\end{align*}
From here, we can establish the following relationships:
\begin{align}
	\Lambda_{t,t} &= 1 \label{eq:Firms.Stochastic.Discount.Current}
	\\
	\Lambda_{t+1,t+1+j} & = \beta^j \frac{\lambda_{t+1+j}}{\lambda_{t+1}} \frac{P_{t+1}}{P_{t+1+j}}
	\\
	\Lambda_{t,t+1+j} & = \beta^{j+1} \frac{\lambda_{t+1+j}}{\lambda_{t}} \frac{P_{t}}{P_{t+1+j}} = \beta \frac{\lambda_{t+1}}{\lambda_{t}} \frac{P_{t}}{P_{t+1}} \beta^j \frac{\lambda_{t+1+j}}{\lambda_{t+1}} \frac{P_{t+1}}{P_{t+1+j}} = \beta \frac{\lambda_{t+1}}{\lambda_t} \Pi_{t+1}^{-1} \Lambda_{t+1,t+1+j} \label{eq:Firms.Stochastic.Discount.Relationship}
\end{align}
We will need this later to derive the recursive nonlinear price setting equations.

\item The Lagrangian of the intermediate firm is
\begin{footnotesize}
\begin{align*}
	\pounds^f &= E_t \sum_{j=0}^{\infty}\Lambda_{t,t+j} P_{t+j} \left[\frac{P_{t+j}(f)}{P_{t+j}} y_{t+j}(f) - w_{t+j} l_{d,t+j}(f) + mc_{t+j}(f)\left(a_{t+j} l_{d,t+j}(f) - y_{t+j}(f)\right)\right]
\end{align*}
\end{footnotesize}
\(mc_t(f)\) denotes the Lagrange multiplier which is the shadow price of producing an additional output unit in the optimum;
  obviously, this is our understanding of real marginal costs.
Taking the derivative wrt \(l_{d,t}(f)\) actually boils down to a static problem (as we only need to evaluate for \(j=0\)) and yields:
\begin{align}
	w_t &= mc_t(f) a_t = mc_t(f) \frac{y_t(f)}{l_{d,t}(f)} \label{eq:Firms.Labor.Demand}
\end{align}
where we substituted the production function~\eqref{eq:Firms_ProductionFunction} for \(a_t\).
This is the labor demand function, which implies that the labor-to-output ratio is the same across firms and equal to \(a_t\).
Note that all firms face the same factor prices and all have access to the same production technology \(a_t\);
  hence, from the above equation it is evident that marginal costs are identical across firms
\begin{align}
	mc_t(f) = \frac{w_t}{a_t}\label{eq:Firms.Marginal.Costs}
\end{align}
This means that aggregate marginal costs are also equal to the ratio between the real wage and technology:
\begin{align*}
	mc_t = \int_0^1 mc_t(f) df = \frac{w_t}{a_t}
\end{align*}
	
\item The Lagrangian of the intermediate firm is
\begin{footnotesize}
\begin{align}
	\pounds^f &= E_t \sum_{j=0}^{\infty}\Lambda_{t,t+j} P_{t+j} \left[ {\left(\frac{P_{t+j}(f)}{P_{t+j}}\right)}^{1-\epsilon} y_{t+j} - w_{t+j} l_{d,t+j}(f) + mc_{t+j}(f)\left(a_{t+j} l_{d,t+j}(f) - \left(\frac{P_{t+j}(f)}{P_{t+j}}\right)^{-\epsilon} y_{t+j}\right)\right]
		\label{eq:Firms.Lagrangian}
\end{align}
\end{footnotesize}
where (compared to above) we used the demand curve~\eqref{eq:firms_demand} to substitute for \(y_t(f)\).
When firms decide how to set their price they need to take into account that due to the Calvo mechanism
  they might get stuck at \(\widetilde{P}_t(f)\) for a number of periods \(j=1,2,\ldots \) before they can re-optimize again.
The probability of such a situation is \(\theta^j\).
Therefore, when firms are able to change prices in period \(t\), they take this into account and the above Lagrangian of the expected discounted sum of nominal profits becomes:
\begin{align}
	\pounds^{f^c} &= E_t \sum_{j=0}^{\infty}\theta^j \Lambda_{t,t+j} P_{t+j}\left[ \left(\frac{\widetilde{P}_{t}(f)}{P_{t+j}}\right)^{1-\epsilon} y_{t+j} - w_{t+j} l_{d,t+j}(f) + mc_{t+j}\left(a_{t+j} l_{d,t+j}(f) - {\left(\frac{\widetilde{P}_{t}(f)}{P_{t+j}}\right)}^{-\epsilon} y_{t+j}\right)\right]
	\\
	&= E_t \sum_{j=0}^{\infty}\theta^j \Lambda_{t,t+j} P_{t+j}^\epsilon y_{t+j} \left[ \widetilde{P}_{t}(f)^{1-\epsilon}  - P_{t+j} \cdot mc_{t+j} \cdot \widetilde{P}_{t}(f)^{-\epsilon} \right] + \ldots
	\label{eq:Firms.Lagrangian.Calvo}
\end{align}
where in the second line we focus only on relevant parts for the optimization wrt to \(\widetilde{P}_t(f)\).
Moreover, we took into account that \(mc_t(f) = mc_t\).
		
The first-order condition of maximizing \(\pounds^{f^c}\) wrt to \(\widetilde{P}_t(f)\) is
\begin{align}
	0= E_t \sum_{j=0}^{\infty}\theta^j \Lambda_{t,t+j} P_{t+j}^\epsilon y_{t+j} \left[ (1-\epsilon)\cdot \widetilde{P}_{t}(f)^{-\epsilon}  +\epsilon \cdot P_{t+j} \cdot mc_{t+j} \widetilde{P}_{t}(f)^{-\epsilon-1}\right]
\end{align}
As \(\widetilde{P}_t(f)>0\) does not depend on \(j\), we multiply by \({\widetilde{P}_t(f)}^{\epsilon+1}\):
\begin{align}
	0= E_t \sum_{j=0}^{\infty}\theta^j \Lambda_{t,t+j} P_{t+j}^{\epsilon} y_{t+j} \left[ (1-\epsilon)\cdot\widetilde{P}_t(f) +\epsilon \cdot P_{t+j} \cdot mc_{t+j}  \right]
\end{align}
Rearranging
\begin{align}
	\widetilde{P}_t(f) \cdot E_t \sum_{j=0}^{\infty}\theta^j \Lambda_{t,t+j} P_{t+j}^{\epsilon} y_{t+j}  = \frac{\epsilon}{\epsilon-1} \cdot E_t \sum_{j=0}^{\infty}\theta^j \Lambda_{t,t+j} P_{t+j}^{\epsilon+1} y_{t+j} mc_{t+j}
\end{align}
Dividing both sides by \(P_t^{\epsilon+1}\)
\begin{align}
	\underbrace{\frac{\widetilde{P}_t(f)}{P_t}}_{\widetilde{p}_t} \cdot  \underbrace{E_t\sum_{j=0}^{\infty}\theta^j \Lambda_{t,t+j} \left(\frac{P_{t+j}}{P_t}\right)^{\epsilon} y_{t+j}}_{S_{1,t}}  = \frac{\epsilon}{\epsilon-1} \cdot \underbrace{E_t \sum_{j=0}^{\infty}\theta^j \Lambda_{t,t+j} {\left(\frac{P_{t+j}}{P_t}\right)}^{\epsilon+1} y_{t+j} mc_{t+j}}_{S_{2,t}}
\end{align}
Note that all firms that reset prices face the same problem and therefore set the same price, \(\widetilde{P}_t(f) =\widetilde{P}_t\).
This is also evident by looking at the infinite sums, \(S_{1,t}\) and \(S_{2,t}\), because these do not depend on \(f\).
Therefore, we can drop the \(f\) in \(\widetilde{P}_t(f)\) and define \(\widetilde{p}_t:= \frac{\widetilde{P}_t}{P_t}\).
The first-order condition can thus be written compactly:
\begin{align}
	\widetilde{p}_t \cdot S_{1,t} = \frac{\epsilon}{\epsilon-1} \cdot S_{2,t}
\end{align}
		
Moreover, the two infinite sums can be written recursively.
For this we make use of the relationships for the stochastic discount factor~\eqref{eq:Firms.Stochastic.Discount.Current} and~\eqref{eq:Firms.Stochastic.Discount.Relationship}.
The first recursive sum can be written as:
\begin{align*}
	S_{1,t} &= 
	E_t\sum_{j=0}^{\infty}\theta^j \Lambda_{t,t+j} \left(\frac{P_{t+j}}{P_t}\right)^{\epsilon} y_{t+j}
	\\
	&= y_t + E_t\sum^{\infty}_{\color{red}{j=1}}\theta^{\color{red}{j}} \Lambda_{t,t+\color{red}{j}} {\left(\frac{P_{t+\color{red}{j}}}{P_t}\right)}^{\epsilon} y_{t+\color{red}{j}}
	\\
	&= y_t + E_t\sum^{\infty}_{\color{red}{j=0}}\theta^{\color{red}{j+1}} \Lambda_{t,t+\color{red}{j+1}} {\left(\frac{P_{t+\color{red}{j+1}}}{P_t}\right)}^{\epsilon} y_{t+\color{red}{j+1}}
	\\
	&= y_t + E_t\sum^{\infty}_{j=0}\theta^{j+1} {\color{blue}{\Lambda_{t,t+j+1}}} {\left({\color{green}{\frac{P_{t+j+1}}{P_{t+1}}\frac{P_{t+1}}{P_{t}}}}\right)}^{\epsilon} y_{t+j+1}
	\\
	&= y_t + E_t\sum^{\infty}_{j=0}\theta^{j+1} {\color{blue}{\beta \frac{\lambda_{t+1}}{\lambda_t}\Pi_{t+1}^{-1} \Lambda_{t+1,t+1+j}}} {\left({\color{green}{\frac{P_{t+j+1}}{P_{t+1}}\Pi_{t+1}}}\right)}^{\epsilon} y_{t+j+1}
	\\
	&= y_t + \theta\beta E_t\frac{\lambda_{t+1}}{\lambda_t}\Pi_{t+1}^{\epsilon-1} \underbrace{E_t\sum^{\infty}_{j=0}\theta^{j} \Lambda_{t+1,t+1+j} {\left(\frac{P_{t+j+1}}{P_{t+1}}\right)}^{\epsilon} y_{t+j+1}}_{=S_{1,t+1}}
\end{align*}
The second recursive sum can be written as
\begin{align*}
	S_{2,t} &= 
	E_t\sum_{j=0}^{\infty}\theta^j \Lambda_{t,t+j} {\left(\frac{P_{t+j}}{P_t}\right)}^{\epsilon+1} y_{t+j} mc_{t+j}
	\\
	&= y_t mc_t + E_t\sum^{\infty}_{\color{red}{j=1}}\theta^{\color{red}{j}} \Lambda_{t,t+\color{red}{j}} {\left(\frac{P_{t+\color{red}{j}}}{P_t}\right)}^{\epsilon+1} y_{t+\color{red}{j}} mc_{t+\color{red}{j}}
	\\
	&= y_t mc_t + E_t\sum^{\infty}_{\color{red}{j=0}}\theta^{\color{red}{j+1}} \Lambda_{t,t+\color{red}{j+1}} {\left(\frac{P_{t+\color{red}{j+1}}}{P_t}\right)}^{\epsilon+1} y_{t+\color{red}{j+1}} mc_{t+\color{red}{j+1}}
	\\
	&= y_t mc_t + E_t\sum^{\infty}_{j=0}\theta^{j+1} {\color{blue}{\Lambda_{t,t+j+1}}} {\left({\color{green}{\frac{P_{t+j+1}}{P_{t+1}}\frac{P_{t+1}}{P_{t}}}}\right)}^{\epsilon+1} y_{t+j+1} mc_{t+j+1}
	\\
	&= y_t mc_t + E_t\sum^{\infty}_{j=0}\theta^{j+1} {\color{blue}{\beta \frac{\lambda_{t+1}}{\lambda_t}\Pi_{t+1}^{-1}\Lambda_{t+1,t+1+j}}} {\left({\color{green}{\frac{P_{t+j+1}}{P_{t+1}}\Pi_{t+1}}}\right)}^{\epsilon+1} y_{t+j+1} mc_{t+j+1}
	\\
	&= y_t mc_t + 
	\theta\beta E_t\frac{\lambda_{t+1}}{\lambda_t}\Pi_{t+1}^{\epsilon} \underbrace{E_t\sum^{\infty}_{j=0}\theta^{j} \Lambda_{t+1,t+1+j} {\left(\frac{P_{t+j+1}}{P_{t+1}}\right)}^{\epsilon+1} y_{t+j+1} mc_{t+j+1}}_{=S_{2,t+1}}
	\end{align*}
To sum up:
\begin{align}
	S_{1,t} &= y_t + \theta \beta E_t \frac{\lambda_{t+1}}{\lambda_{t}} \Pi_{t+1}^{\epsilon-1} S_{1,t+1}\\
	S_{2,t} &= y_t mc_t + \theta \beta E_t \frac{\lambda_{t+1}}{\lambda_{t}} \Pi_{t+1}^{\epsilon} S_{2,t+1}
\end{align}
\item The law of motion for \(\widetilde{p}_t=\frac{\widetilde{P}_t}{P_t}\) is given by the aggregate price index~\eqref{eq:Firms_DS_Pricing} which can be re-arranged to
\begin{align}
	1 &= \int_{0}^{1} {\left(\frac{P_t(f)}{P_t}\right)}^{1-\epsilon}df \label{eq:Aggregate.Price}
\end{align}
Due to the Calvo mechanism we get that \((1-\theta)\) firms can re-set their price to \(\widetilde{P}_t\),
  whereas the remaining \(\theta \) firms cannot and set their price equal to \(P_{t-1}\).
Therefore:
\begin{align}
	1 &= \int_{optimizers} {\left(\frac{P_t(f)}{P_t}\right)}^{1-\epsilon} df  + \int_{non-optimizers} {\left(\frac{P_t(f)}{P_t}\right)}^{1-\epsilon} df \\
	1&= (1-\theta) {\left(\frac{\widetilde{P}_t}{P_t}\right)}^{1-\epsilon} + \theta \int_{0}^1 {\left(\frac{P_{t-1}(f)}{P_t}{\color{red}{\frac{P_{t-1}}{P_{t-1}}}}\right)}^{1-\epsilon}df\\
	1&= (1-\theta) \widetilde{p}_t^{1-\epsilon} + \theta {\left(\frac{P_{t-1}}{P_{t}}\right)}^{1-\epsilon} \int_{0}^1 {\left(\frac{P_{t-1}(f)}{P_{t-1}}\right)}^{1-\epsilon}df\\
	1&=(1-\theta) \widetilde{p}_t^{1-\epsilon}  + \theta \Pi_t^{1-\epsilon} \underbrace{\int_{0}^{1} {\left(\frac{P_{t-1}(f)}{P_{t-1}} \right)}^{1-\epsilon}df}_{\overset{\eqref{eq:Aggregate.Price}}{=}1}\\
	1&=(1-\theta) \widetilde{p}_t^{1-\epsilon}  + \theta \Pi_t^{\epsilon-1}
\end{align}
	
\item Private bonds \(B_t\) are in zero net supply on the budget constraint.
Note that this condition can only be imposed after taking first order conditions.
It would be invalid to eliminate bonds already in the budget constraint of the household.
Even if bonds are in zero net supply, households savings behavior in equilibrium still needs to be consistent with the bond market clearing.
	
\item In an equilibrium, labor demand from the intermediate firms needs to be equal to the labor supply of the households; hence:
\begin{align}
	\int_{0}^1 l_{d,t}(f) df = l_t
\end{align}
so \(l_t\) denotes equilibrium hours worked (both supplied and demanded).

\item Given the demand for good \(y_t(f)\) and the Dixit-Stiglitz aggregation technology, we get:
\begin{align*}
	\int_0^1 y_t(f) P_t(f) df = \int_0^1 {\left(\frac{P_t(f)}{P_t}\right)}^{-\epsilon} y_t P_t(f) df = P_t y_t \underbrace{\int_0^1 {\left(\frac{P_t(f)}{P_t}\right)}^{1-\epsilon}  df}_{\overset{\eqref{eq:AggregatePriceIndex}}{=1}} = P_t y_t
\end{align*}
Moreover, from the labor market we have \(l_t = \int_{0}^1 l_{d,t}(f) df\).
Plugging both expressions into aggregate real profits:
\begin{align*}
	div_t = \int_0^1 div_t(f) df = \int_0^1 \frac{P_t(f)}{P_t} y_t(f) df - \int_0^1 w_t l_{d,t}(f) df = y_t - w_t l_t
\end{align*}

\item Revisit the budget constraint in real terms:
\begin{align*}
	\int_0^1 \frac{P_t(h)}{P_t} c_t(h) dh + Q_t b_t \leq b_{t-1}\Pi_t^{-1} + w_t l_t + \int_0^1 div_t(f) df
\end{align*}
which becomes
\begin{align*}
	c_t = w_t l_t + (y_t - w_t l_t) = y_t
\end{align*}
in an optimal allocation with cleared markets. This is the aggregate demand equation.
	
\item Define \(y_t^{sum} = \int_{0}^1 y_t(f) df\).
Using the production function~\eqref{eq:Firms_ProductionFunction} and labor market clearing we get:
\begin{align}
	y_t^{sum} = \int_{0}^1 a_t l_{d,t}(f) df = a_t l_t
\end{align}
Furthermore, due to the demand for intermediate good \(y_t(f)\) in \eqref{eq:firms_demand} we get:
\begin{align}
	y_t^{sum} = y_t \underbrace{\int_{0}^1 {\left(\frac{P_t(f)}{P_t}\right)}^{-\epsilon} df}_{=p_t^*}
\end{align}
Equating both yields:
\begin{align}
	p_t^* y_t =  a_t l_t
\end{align}
This is the aggregate supply equation.
Price frictions, however, imply that resources will not be efficiently allocated
  as prices are too high because not all firms can re-optimize their price in every period.
This inefficiency is measured by \(p_t^*<1\).

\item The law of motion for the efficiency distortion \(p_t^*\) is given due to the Calvo price mechanism, i.e.:
\begin{align*}
	p_t^* &= \int_0^1{\left(\frac{P_t(f)}{P_t}\right)}^{-\epsilon} df\\
	p_t^* &= \int_{optimizers} {\left(\frac{P_t(f)}{P_t}\right)}^{-\epsilon} df + \int_{non-optimizers}{\left(\frac{P_t(f)}{P_t}\right)}^{-\epsilon} df\\
	p_t^* & = (1-\theta) \widetilde{p}_t^{-\epsilon} + \theta \int_0^1 {\left(\frac{P_{t-1}(f)}{P_t}\right)}^{-\epsilon} df\\
	p_t^* & = (1-\theta) \widetilde{p}_t^{-\epsilon} + \theta \int_0^1 {\left(\frac{P_{t-1}(f)}{P_t }\frac{P_{t-1}}{P_{t-1}}\right)}^{-\epsilon} df\\
	p_t^* & = (1-\theta) \widetilde{p}_t^{-\epsilon} + \theta {\left(\frac{P_{t-1}}{P_{t}}\right)}^{-\epsilon} \int_0^1 {\left(\frac{P_{t-1}(f)}{P_{t-1} }\right)}^{-\epsilon} df\\
	p_t^* & = (1-\theta) \widetilde{p}_t^{-\epsilon} + \theta \Pi_t^{\epsilon} \underbrace{\int_0^1 {\left(\frac{P_{t-1}(f)}{P_{t-1} }\right)}^{-\epsilon} df}_{=p_{t-1}^*}\\
	p_t^* & = (1-\theta) \widetilde{p}_t^{-\epsilon} + \theta \Pi_t^{\epsilon} p_{t-1}^*
\end{align*}

\end{enumerate}