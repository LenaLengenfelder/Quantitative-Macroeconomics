\begin{enumerate}

\item \textbf{Geometric sequence:} First, note that we can use the geometric sequence with or without the lag operator, i.e.
\begin{align*}
{(1-\phi)}^{-1} &= \lim\limits_{j\rightarrow \infty}(\phi^0 + \phi^1 + \phi^2 + \cdots + \phi^j) = \sum_{j=0}^\infty \phi^j
\\
{(1-\phi L)}^{-1} &= \lim\limits_{j\rightarrow \infty}({(\phi L)}^0 + {(\phi L)}^1 + {(\phi L)}^2 + \cdots + {(\phi L)}^j) = \sum_{j=0}^\infty {(\phi L)}^j
\end{align*}
The proof for this is pretty simple. Denote
\begin{align*}
S_{k} =	\sum_{j=0}^{k} \phi^j =	1 + \phi^1 + \phi^2 + \phi^3 + \cdots + \phi^k
\end{align*}
then multiply with \(\phi\):
\begin{align*}
\phi S_k = \phi^1 + \phi^2 + \phi^3 + \cdots + \phi^{k+1}
\end{align*}
Now look at \(S_k - \phi S_k = (1 - \phi) S_{k}\):
\begin{align*}
(1-\phi) S_{k} &= 1 - \phi^{k+1}
\Leftrightarrow
S_{k}	= \frac{1}{1 - \phi} - \frac{\phi^{k+1}}{1-\phi}
\end{align*} 
Looking at the limit of \(S_{k}\) for \(k \to \infty \), we get
\[\lim\limits_{k \to \infty} S_{k} = \frac{1}{1 - \phi}\]
\\
Next let's get a representation of the process \(y_t\) that is useful to compute the moments.

We can do this in different ways:
    \begin{itemize}

    \item \textbf{Recursive substitution} (starting at some infinite time \(j\)):
    \begin{align*}
    y_t &= c + \phi y_{t-1} + \varepsilon_t\\
        &= c + \phi \left( c + \phi y_{t-2} + \varepsilon_{t-1}\right) + \varepsilon_t\\ 
        &= c + \phi c + \varepsilon_t + \phi \varepsilon_{t-1} + \phi^2( c+ \phi y_{t-3} + \varepsilon_{t-2} )\\
        &\vdots\\
        & = c + \phi c + \phi^2 c^2 + \cdots + \phi^j c^j 
        + \varepsilon_t + \phi \varepsilon_{t-1} + \phi^2 \varepsilon_{t-2} + \cdots + \phi^j \varepsilon_{t-j}
        + \phi^{j+1} y_{t-{j+1}}
    \end{align*}
    \(y_t\) is a linear function of an initial value \(\phi^{j+1} y_{t-{j+1}}\),
      historical values of the white noise process \(\varepsilon_t\), and a sum of polynomials in \(c\).
    If \(|\phi|<1\) and \(j\) becomes large,
      then \( \phi^{j+1} y_{t-{j+1}} \rightarrow 0\),
      thus we get a so-called \(MA(\infty)\) process:
    \begin{align*}
    y_t &= \underbrace{c + \phi c + \phi^2 c^2 + \ldots}_{\sum_{j=0}^\infty \phi^j c^j} + \underbrace{\varepsilon_t + \phi \varepsilon_{t-1} + \phi^2 \varepsilon_{t-2} + \ldots}_{\sum_{j=0}^\infty \phi^j \varepsilon_{t-j}}
    \\
    &= c\sum_{j=0}^\infty \phi^j + \sum_{j=0}^\infty \phi^j \varepsilon_{t-j}    
    = \frac{c}{1-\phi} + \sum_{j=0}^\infty \phi^j \varepsilon_{t-j}
    \end{align*}
    

    \item \textbf{With Lag Operators}: works only if \(|\phi| < 1\) and \(\{y_t\}\) is bounded (that is, there exists a finite number \(k\) such that \(|y_t| < k\) for all \(t\)).
    Then
    \begin{align*}
    (1-\phi L) y_t = c + \varepsilon_t\\
    (1-\phi L)^{-1}(1-\phi L) y_t =	y_t = (1-\phi L)^{-1} c + (1-\phi L)^{-1}\varepsilon_t\\
    \end{align*}
    Using the geometric series, we get:
    \begin{align*}
    y_t &= (1+\phi L + \phi^2 L^2+\cdots+(\phi L)^j) c
        + (1+\phi L + \phi^2 L^2+\cdots+(\phi L)^j)\varepsilon_t
\\
        &= \left(c + \phi c + \phi^2 c + \ldots \right) + \left(\varepsilon_t + \phi \varepsilon_{t-1} + \phi^2 \varepsilon_{t-2}+\ldots\right)
        = c \sum_{j=0}^\infty \phi^j + \sum_{j=0}^\infty \phi^j \varepsilon_{t-j}
        \\
        &= \frac{c}{1-\phi} + \sum_{j=0}^\infty \phi^j \varepsilon_{t-j}
    \end{align*}
    \end{itemize}
    If we can express an AR process as a MA process, we call this process invertible.
    Let's now compute the moments from the MA(\(\infty\)) representation
      and using the fact that \(\varepsilon_t\) is a white noise process:
    \begin{itemize}
      \item \textbf{Unconditional Mean:}
      \begin{align*}
      E[y_t] &= E\left[\frac{c}{1-\phi}\right] + E\left[\sum_{j=1}^{\infty} \phi^j \varepsilon_{t-j}\right] = \frac{c}{1-\phi} \sum_{j=1}^{\infty} \phi^j E[\varepsilon_{t-j}]
      \\
      &= \underbrace{\frac{c}{1-\phi}}_{:=\mu}
      \end{align*}
      As this process is covariance-stationary, the unconditional mean is time invariant.
      We typically denote this time-independence by using the greek letter $\mu$.
      \item \textbf{Unconditional variance:}
      \begin{align*}
      Var[y_t] &= E\left[(y_t -E[y_t])(y_t - E[y_t])\right] = E\left[\left(\sum_{j=0}^\infty \phi^j \varepsilon_{t-j}\right) \left(\sum_{j=0}^\infty \phi^j \varepsilon_{t-j}\right) \right]
      \\
      &= E\left[ \phi^0 \phi^0 \varepsilon_{t} \varepsilon_{t} + \phi^0 \phi^1 \varepsilon_{t} \varepsilon_{t-1} + \phi^0 \phi^2 \varepsilon_{t} \varepsilon_{t-2} + \phi^0 \phi^3 \varepsilon_{t} \varepsilon_{t-3} + \ldots \right.
      \\
      & \qquad~\phi^1 \phi^0 \varepsilon_{t-1} \varepsilon_{t} + \phi^1 \phi^1 \varepsilon_{t-1} \varepsilon_{t-1} + \phi^1 \phi^2 \varepsilon_{t-1} \varepsilon_{t-2} + \phi^1 \phi^3 \varepsilon_{t-1} \varepsilon_{t-3} + \ldots
      \\
      & \qquad~\phi^2 \phi^0 \varepsilon_{t-2} \varepsilon_{t} + \phi^2 \phi^1 \varepsilon_{t-2} \varepsilon_{t-1} + \phi^2 \phi^2 \varepsilon_{t-2} \varepsilon_{t-2} + \phi^2 \phi^3 \varepsilon_{t-2} \varepsilon_{t-3} + \ldots
      \\&\left.\qquad~\ldots\right]
      \\
      &= \phi^0 \phi^0 E[\varepsilon_{t} \varepsilon_{t}] + \phi^0 \phi^1 E[\varepsilon_{t} \varepsilon_{t-1}] + \phi^0 \phi^2 E[\varepsilon_{t} \varepsilon_{t-2}] + \phi^0 \phi^3 E[\varepsilon_{t} \varepsilon_{t-3}] + \ldots
      \\
      & \qquad~\phi^1 \phi^0 E[\varepsilon_{t-1} \varepsilon_{t}] + \phi^1 \phi^1 E[\varepsilon_{t-1} \varepsilon_{t-1}] + \phi^1 \phi^2 E[\varepsilon_{t-1} \varepsilon_{t-2}] + \phi^1 \phi^3 E[\varepsilon_{t-1} \varepsilon_{t-3}] + \ldots
      \\
      & \qquad~\phi^2 \phi^0 E[\varepsilon_{t-2} \varepsilon_{t}] + \phi^2 \phi^1 E[\varepsilon_{t-2} \varepsilon_{t-1}] + \phi^2 \phi^2 E[\varepsilon_{t-2} \varepsilon_{t-2}] + \phi^2 \phi^3 E[\varepsilon_{t-2} \varepsilon_{t-3}] + \ldots
      \\&\qquad~\ldots
      \end{align*}
      Note that \(\varepsilon_t\) is a white-noise process with variance \(E[\varepsilon_{t-j} \varepsilon_{t-j}]=\sigma_\varepsilon^2\) for any \(j\),
      but zero autocovariance, i.e. \(E[\varepsilon_{t-j} \varepsilon_{t-k}]=0\) for any \(j \neq k\).
      Therefore:
      \begin{align*}
      Var[y_t] &= \phi^0 \phi^0 E[\varepsilon_{t} \varepsilon_{t}] + \phi^1 \phi^1 E[\varepsilon_{t-1} \varepsilon_{t-1}] + \phi^2 \phi^2 E[\varepsilon_{t-2} \varepsilon_{t-2}] + \phi^3 \phi^3 E[\varepsilon_{t-3} \varepsilon_{t-3}] + \ldots
      \\
      &= \sum_{j=0}^\infty (\phi^{2})^j E[\varepsilon_{t-j} \varepsilon_{t-j}]
      = \sum_{j=0}^\infty (\phi^{2})^j \sigma_\varepsilon^2
      = \sigma_\varepsilon^2 \sum_{j=0}^\infty (\phi^{2})^j
      \\
      =& \underbrace{\sigma_\varepsilon^2 \frac{1}{1-\phi^2}}_{:= \gamma_0}
      \end{align*}
      As the process is covariance-stationary, the unconditional variance is time invariant.
      We typically denote this time-independence by using the Greek letter \(\gamma_0\).
      \item \textbf{Unconditional autocovariance:}
      \begin{align*}
      Var[y_t,y_{t-k}] &= E\left[(y_t -E[y_t])(y_t - E[y_{t-k}])\right] = E\left[\left(\sum_{j=0}^\infty \phi^j \varepsilon_{t-j}\right) \left(\sum_{j=0}^\infty \phi^j \varepsilon_{t-k-j}\right) \right]
      \\
      &= E\left[ \phi^0 \phi^0 \varepsilon_{t} \varepsilon_{t-k} + \phi^0 \phi^1 \varepsilon_{t} \varepsilon_{t-k-1} + \phi^0 \phi^2 \varepsilon_{t} \varepsilon_{t-k-2} + \phi^0 \phi^3 \varepsilon_{t} \varepsilon_{t-k-3} + \ldots \right.
      \\
      & \qquad~\phi^1 \phi^0 \varepsilon_{t-1} \varepsilon_{t-k} + \phi^1 \phi^1 \varepsilon_{t-1} \varepsilon_{t-k-1} + \phi^1 \phi^2 \varepsilon_{t-1} \varepsilon_{t-k-2} + \phi^1 \phi^3 \varepsilon_{t-1} \varepsilon_{t-k-3} + \ldots
      \\
      & \qquad~\phi^2 \phi^0 \varepsilon_{t-2} \varepsilon_{t-k} + \phi^2 \phi^1 \varepsilon_{t-2} \varepsilon_{t-k-1} + \phi^2 \phi^2 \varepsilon_{t-2} \varepsilon_{t-k-2} + \phi^2 \phi^3 \varepsilon_{t-2} \varepsilon_{t-k-3} + \ldots
      \\&\left.\qquad~\ldots\right]      
      \end{align*}
      This can be simplified due to the white noise property of \(\varepsilon_t\) to:
      \begin{align*}
      Var[y_t,y_{t-k}] &= \phi^k (\phi^0 E[\varepsilon_{t-k} \varepsilon_{t-k}] + \phi^2 E[\varepsilon_{t-k-1} \varepsilon_{t-k-1}]+ \phi^4 E[\varepsilon_{t-k-2} \varepsilon_{t-k-2}] + \ldots )
      \\
      &= \phi^k \sum_{j=0}^\infty (\phi^{2})^j \sigma_\varepsilon^2 = \phi^k \frac{\sigma_\varepsilon^2}{1-\phi^2}      
      = \underbrace{\phi^k \gamma_0}_{:=\gamma_k}
      \end{align*}
      As the process is covariance-stationary, the unconditional autocovariance is only dependent on the time difference \(k\).
      We typically denote this by using the Greek letter \(\gamma_k\).
\end{itemize}
	 	
\item Here is a possible run-script:
\lstinputlisting[style=Matlab-editor,basicstyle=\mlttfamily,title=\lstname]{progs/matlab/acfPlots_run.m}
and the corresponding \texttt{acfPlots.m} function:
\lstinputlisting[style=Matlab-editor,basicstyle=\mlttfamily,title=\lstname]{progs/matlab/acfPlots.m}
\end{enumerate}