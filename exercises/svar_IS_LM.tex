\section[How Well Does the IS-LM Model Fit Postwar US Data]{How Well Does the IS-LM Model Fit Postwar US Data?\label{ex:svarISLM}}
Consider a quarterly model for \(y_t = (\Delta gnp_t, \Delta i_t, i_t-\Delta p_t, \Delta m_t - \Delta p_t)'\),
  where \(gnp_t\) denotes the log of GNP, \(i_t\) the nominal yield on three-month Treasury Bills,
  \(\Delta m_t\) the growth in M1 and \(\Delta p_t\) the inflation rate in the CPI.
There are four shocks in the system: an aggregate supply (AS), a money supply (MS), a money demand (MD) and an aggregate demand (IS) shock.
Ignoring the lagged dependent variables for \textbf{expository} purposes (\(B_1=\cdots =B_p=0\)),
  the unrestricted structural VAR model can be simply written as \(B_0 y_t = \varepsilon_t\). That is:
\begin{align}
\Delta gnp_t &= -b_{12}\Delta i_t -b_{13}(i_t-\Delta p_t) -b_{14}(\Delta m_t-\Delta p_t) + \varepsilon_t^{AS} \label{eq:AS}\\
\Delta i_t &= -b_{21}\Delta gnp_t -b_{23}(i_t-\Delta p_t) -b_{24}(\Delta m_t-\Delta p_t) + \varepsilon_t^{MS} \label{eq:MS}\\
i_t - \Delta p_t &= -b_{31}\Delta gnp_t -b_{32}\Delta i_t -b_{34}(\Delta m_t-\Delta p_t) + \varepsilon_t^{MD} \label{eq:MD}\\
\Delta m_t - \Delta p_t &= -b_{41}\Delta gnp_t -b_{42}\Delta i_t - b_{43} (i_t-\Delta p_t) + \varepsilon_t^{IS} \label{eq:IS}
\end{align}
where \(b_{ij}\) denotes the \(ij\)th element of \(B_0\).
Consider the following identification restrictions:
\begin{itemize}
	\item Money supply shocks do not have contemporaneous effects on output growth, i.e.
	$$\frac{\partial \Delta gnp_t}{\partial \varepsilon_t^{MS}}=0$$	
	\item Money demand shocks do not have contemporaneous effects on output growth, i.e.
	$$\frac{\partial \Delta gnp_t}{\partial \varepsilon_t^{MD}}=0$$	
	\item Monetary authority does not react contemporaneously to changes in the price level.\\Hint: compute from equation \eqref{eq:MS}:
	$$\frac{\partial \Delta i_t}{\partial \Delta p_t}=0$$
	\item Money supply shocks, money demand shocks and aggregate demand shocks do not have long-run effects on the log of real GNP:	
	$$\frac{\partial gnp_t}{\partial \varepsilon_t^{MS}}=0,\qquad \frac{\partial gnp_t}{\partial \varepsilon_t^{MD}}=0,\qquad \frac{\partial gnp_t}{\partial \varepsilon_t^{IS}}=0$$	
	\item The structural shocks are uncorrelated with covariance matrix $E(\varepsilon_t \varepsilon_t')=\Sigma_\varepsilon$.
	In other words, the variances are \textbf{not} normalized.
\end{itemize}
Solve the following exercises:
\begin{enumerate}
	\item Derive the implied exclusion restrictions on the matrices $B_0$, $B_0^{-1}$ and $\Theta(1)$.
	\item Consider data given in the csv file \texttt{gali1992.csv}.
	Estimate a VAR(4) model with a constant.
	\item Estimate the structural impact matrix using a nonlinear equation solver,
	  i.e.\ the objective is to find the unknown elements of $B_0^{-1}$ and the diagonal elements of $\Sigma_\varepsilon$ such that
	$$\begin{bmatrix}
	vech(B_0^{-1} \Sigma_\varepsilon B_0^{-1'}-\hat{\Sigma}_u)\\
	\text{short-run restrictions on }B_0 \text{ and } B_0^{-1} \\
	\text{long-run restrictions on }\Theta(1)\\
	\end{bmatrix}$$
	is minimized.
	Normalize the shocks such that the diagonal elements of $B_0^{-1}$ are positive.
	\item Use the implied estimates of $B_0^{-1}$ and $\Sigma_\varepsilon$ to plot the structural impulse responses functions
	for (i) real GNP, (ii) the yield on Treasury Bills, (iii) the real interest rate and (iv) real money growth.
    Add 68\% and 95\% confidence intervals using a bootstrap approach.
\end{enumerate}

\paragraph{Readings}
\begin{itemize}
	\item \textcite{Gali_1992_HowWellDoes}
\end{itemize}

\begin{solution}\textbf{Solution to \nameref{ex:svarISLM}}
\ifDisplaySolutions
\section[How Well Does the IS-LM Model Fit Postwar US Data]{How Well Does the IS-LM Model Fit Postwar US Data?\label{ex:svarISLM}}
Consider a quarterly model for \(y_t = (\Delta gnp_t, \Delta i_t, i_t-\Delta p_t, \Delta m_t - \Delta p_t)'\),
  where \(gnp_t\) denotes the log of GNP, \(i_t\) the nominal yield on three-month Treasury Bills,
  \(\Delta m_t\) the growth in M1 and \(\Delta p_t\) the inflation rate in the CPI.
There are four shocks in the system: an aggregate supply (AS), a money supply (MS), a money demand (MD) and an aggregate demand (IS) shock.
Ignoring the lagged dependent variables for \textbf{expository} purposes (\(B_1=\cdots =B_p=0\)),
  the unrestricted structural VAR model can be simply written as \(B_0 y_t = \varepsilon_t\). That is:
\begin{align}
\Delta gnp_t &= -b_{12}\Delta i_t -b_{13}(i_t-\Delta p_t) -b_{14}(\Delta m_t-\Delta p_t) + \varepsilon_t^{AS} \label{eq:AS}\\
\Delta i_t &= -b_{21}\Delta gnp_t -b_{23}(i_t-\Delta p_t) -b_{24}(\Delta m_t-\Delta p_t) + \varepsilon_t^{MS} \label{eq:MS}\\
i_t - \Delta p_t &= -b_{31}\Delta gnp_t -b_{32}\Delta i_t -b_{34}(\Delta m_t-\Delta p_t) + \varepsilon_t^{MD} \label{eq:MD}\\
\Delta m_t - \Delta p_t &= -b_{41}\Delta gnp_t -b_{42}\Delta i_t - b_{43} (i_t-\Delta p_t) + \varepsilon_t^{IS} \label{eq:IS}
\end{align}
where \(b_{ij}\) denotes the \(ij\)th element of \(B_0\).
Consider the following identification restrictions:
\begin{itemize}
	\item Money supply shocks do not have contemporaneous effects on output growth, i.e.
	$$\frac{\partial \Delta gnp_t}{\partial \varepsilon_t^{MS}}=0$$	
	\item Money demand shocks do not have contemporaneous effects on output growth, i.e.
	$$\frac{\partial \Delta gnp_t}{\partial \varepsilon_t^{MD}}=0$$	
	\item Monetary authority does not react contemporaneously to changes in the price level.\\Hint: compute from equation \eqref{eq:MS}:
	$$\frac{\partial \Delta i_t}{\partial \Delta p_t}=0$$
	\item Money supply shocks, money demand shocks and aggregate demand shocks do not have long-run effects on the log of real GNP:	
	$$\frac{\partial gnp_t}{\partial \varepsilon_t^{MS}}=0,\qquad \frac{\partial gnp_t}{\partial \varepsilon_t^{MD}}=0,\qquad \frac{\partial gnp_t}{\partial \varepsilon_t^{IS}}=0$$	
	\item The structural shocks are uncorrelated with covariance matrix $E(\varepsilon_t \varepsilon_t')=\Sigma_\varepsilon$.
	In other words, the variances are \textbf{not} normalized.
\end{itemize}
Solve the following exercises:
\begin{enumerate}
	\item Derive the implied exclusion restrictions on the matrices $B_0$, $B_0^{-1}$ and $\Theta(1)$.
	\item Consider data given in the csv file \texttt{gali1992.csv}.
	Estimate a VAR(4) model with a constant.
	\item Estimate the structural impact matrix using a nonlinear equation solver,
	  i.e.\ the objective is to find the unknown elements of $B_0^{-1}$ and the diagonal elements of $\Sigma_\varepsilon$ such that
	$$\begin{bmatrix}
	vech(B_0^{-1} \Sigma_\varepsilon B_0^{-1'}-\hat{\Sigma}_u)\\
	\text{short-run restrictions on }B_0 \text{ and } B_0^{-1} \\
	\text{long-run restrictions on }\Theta(1)\\
	\end{bmatrix}$$
	is minimized.
	Normalize the shocks such that the diagonal elements of $B_0^{-1}$ are positive.
	\item Use the implied estimates of $B_0^{-1}$ and $\Sigma_\varepsilon$ to plot the structural impulse responses functions
	for (i) real GNP, (ii) the yield on Treasury Bills, (iii) the real interest rate and (iv) real money growth.
    Add 68\% and 95\% confidence intervals using a bootstrap approach.
\end{enumerate}

\paragraph{Readings}
\begin{itemize}
	\item \textcite{Gali_1992_HowWellDoes}
\end{itemize}

\begin{solution}\textbf{Solution to \nameref{ex:svarISLM}}
\ifDisplaySolutions
\section[How Well Does the IS-LM Model Fit Postwar US Data]{How Well Does the IS-LM Model Fit Postwar US Data?\label{ex:svarISLM}}
Consider a quarterly model for \(y_t = (\Delta gnp_t, \Delta i_t, i_t-\Delta p_t, \Delta m_t - \Delta p_t)'\),
  where \(gnp_t\) denotes the log of GNP, \(i_t\) the nominal yield on three-month Treasury Bills,
  \(\Delta m_t\) the growth in M1 and \(\Delta p_t\) the inflation rate in the CPI.
There are four shocks in the system: an aggregate supply (AS), a money supply (MS), a money demand (MD) and an aggregate demand (IS) shock.
Ignoring the lagged dependent variables for \textbf{expository} purposes (\(B_1=\cdots =B_p=0\)),
  the unrestricted structural VAR model can be simply written as \(B_0 y_t = \varepsilon_t\). That is:
\begin{align}
\Delta gnp_t &= -b_{12}\Delta i_t -b_{13}(i_t-\Delta p_t) -b_{14}(\Delta m_t-\Delta p_t) + \varepsilon_t^{AS} \label{eq:AS}\\
\Delta i_t &= -b_{21}\Delta gnp_t -b_{23}(i_t-\Delta p_t) -b_{24}(\Delta m_t-\Delta p_t) + \varepsilon_t^{MS} \label{eq:MS}\\
i_t - \Delta p_t &= -b_{31}\Delta gnp_t -b_{32}\Delta i_t -b_{34}(\Delta m_t-\Delta p_t) + \varepsilon_t^{MD} \label{eq:MD}\\
\Delta m_t - \Delta p_t &= -b_{41}\Delta gnp_t -b_{42}\Delta i_t - b_{43} (i_t-\Delta p_t) + \varepsilon_t^{IS} \label{eq:IS}
\end{align}
where \(b_{ij}\) denotes the \(ij\)th element of \(B_0\).
Consider the following identification restrictions:
\begin{itemize}
	\item Money supply shocks do not have contemporaneous effects on output growth, i.e.
	$$\frac{\partial \Delta gnp_t}{\partial \varepsilon_t^{MS}}=0$$	
	\item Money demand shocks do not have contemporaneous effects on output growth, i.e.
	$$\frac{\partial \Delta gnp_t}{\partial \varepsilon_t^{MD}}=0$$	
	\item Monetary authority does not react contemporaneously to changes in the price level.\\Hint: compute from equation \eqref{eq:MS}:
	$$\frac{\partial \Delta i_t}{\partial \Delta p_t}=0$$
	\item Money supply shocks, money demand shocks and aggregate demand shocks do not have long-run effects on the log of real GNP:	
	$$\frac{\partial gnp_t}{\partial \varepsilon_t^{MS}}=0,\qquad \frac{\partial gnp_t}{\partial \varepsilon_t^{MD}}=0,\qquad \frac{\partial gnp_t}{\partial \varepsilon_t^{IS}}=0$$	
	\item The structural shocks are uncorrelated with covariance matrix $E(\varepsilon_t \varepsilon_t')=\Sigma_\varepsilon$.
	In other words, the variances are \textbf{not} normalized.
\end{itemize}
Solve the following exercises:
\begin{enumerate}
	\item Derive the implied exclusion restrictions on the matrices $B_0$, $B_0^{-1}$ and $\Theta(1)$.
	\item Consider data given in the csv file \texttt{gali1992.csv}.
	Estimate a VAR(4) model with a constant.
	\item Estimate the structural impact matrix using a nonlinear equation solver,
	  i.e.\ the objective is to find the unknown elements of $B_0^{-1}$ and the diagonal elements of $\Sigma_\varepsilon$ such that
	$$\begin{bmatrix}
	vech(B_0^{-1} \Sigma_\varepsilon B_0^{-1'}-\hat{\Sigma}_u)\\
	\text{short-run restrictions on }B_0 \text{ and } B_0^{-1} \\
	\text{long-run restrictions on }\Theta(1)\\
	\end{bmatrix}$$
	is minimized.
	Normalize the shocks such that the diagonal elements of $B_0^{-1}$ are positive.
	\item Use the implied estimates of $B_0^{-1}$ and $\Sigma_\varepsilon$ to plot the structural impulse responses functions
	for (i) real GNP, (ii) the yield on Treasury Bills, (iii) the real interest rate and (iv) real money growth.
    Add 68\% and 95\% confidence intervals using a bootstrap approach.
\end{enumerate}

\paragraph{Readings}
\begin{itemize}
	\item \textcite{Gali_1992_HowWellDoes}
\end{itemize}

\begin{solution}\textbf{Solution to \nameref{ex:svarISLM}}
\ifDisplaySolutions
\section[How Well Does the IS-LM Model Fit Postwar US Data]{How Well Does the IS-LM Model Fit Postwar US Data?\label{ex:svarISLM}}
Consider a quarterly model for \(y_t = (\Delta gnp_t, \Delta i_t, i_t-\Delta p_t, \Delta m_t - \Delta p_t)'\),
  where \(gnp_t\) denotes the log of GNP, \(i_t\) the nominal yield on three-month Treasury Bills,
  \(\Delta m_t\) the growth in M1 and \(\Delta p_t\) the inflation rate in the CPI.
There are four shocks in the system: an aggregate supply (AS), a money supply (MS), a money demand (MD) and an aggregate demand (IS) shock.
Ignoring the lagged dependent variables for \textbf{expository} purposes (\(B_1=\cdots =B_p=0\)),
  the unrestricted structural VAR model can be simply written as \(B_0 y_t = \varepsilon_t\). That is:
\begin{align}
\Delta gnp_t &= -b_{12}\Delta i_t -b_{13}(i_t-\Delta p_t) -b_{14}(\Delta m_t-\Delta p_t) + \varepsilon_t^{AS} \label{eq:AS}\\
\Delta i_t &= -b_{21}\Delta gnp_t -b_{23}(i_t-\Delta p_t) -b_{24}(\Delta m_t-\Delta p_t) + \varepsilon_t^{MS} \label{eq:MS}\\
i_t - \Delta p_t &= -b_{31}\Delta gnp_t -b_{32}\Delta i_t -b_{34}(\Delta m_t-\Delta p_t) + \varepsilon_t^{MD} \label{eq:MD}\\
\Delta m_t - \Delta p_t &= -b_{41}\Delta gnp_t -b_{42}\Delta i_t - b_{43} (i_t-\Delta p_t) + \varepsilon_t^{IS} \label{eq:IS}
\end{align}
where \(b_{ij}\) denotes the \(ij\)th element of \(B_0\).
Consider the following identification restrictions:
\begin{itemize}
	\item Money supply shocks do not have contemporaneous effects on output growth, i.e.
	$$\frac{\partial \Delta gnp_t}{\partial \varepsilon_t^{MS}}=0$$	
	\item Money demand shocks do not have contemporaneous effects on output growth, i.e.
	$$\frac{\partial \Delta gnp_t}{\partial \varepsilon_t^{MD}}=0$$	
	\item Monetary authority does not react contemporaneously to changes in the price level.\\Hint: compute from equation \eqref{eq:MS}:
	$$\frac{\partial \Delta i_t}{\partial \Delta p_t}=0$$
	\item Money supply shocks, money demand shocks and aggregate demand shocks do not have long-run effects on the log of real GNP:	
	$$\frac{\partial gnp_t}{\partial \varepsilon_t^{MS}}=0,\qquad \frac{\partial gnp_t}{\partial \varepsilon_t^{MD}}=0,\qquad \frac{\partial gnp_t}{\partial \varepsilon_t^{IS}}=0$$	
	\item The structural shocks are uncorrelated with covariance matrix $E(\varepsilon_t \varepsilon_t')=\Sigma_\varepsilon$.
	In other words, the variances are \textbf{not} normalized.
\end{itemize}
Solve the following exercises:
\begin{enumerate}
	\item Derive the implied exclusion restrictions on the matrices $B_0$, $B_0^{-1}$ and $\Theta(1)$.
	\item Consider data given in the csv file \texttt{gali1992.csv}.
	Estimate a VAR(4) model with a constant.
	\item Estimate the structural impact matrix using a nonlinear equation solver,
	  i.e.\ the objective is to find the unknown elements of $B_0^{-1}$ and the diagonal elements of $\Sigma_\varepsilon$ such that
	$$\begin{bmatrix}
	vech(B_0^{-1} \Sigma_\varepsilon B_0^{-1'}-\hat{\Sigma}_u)\\
	\text{short-run restrictions on }B_0 \text{ and } B_0^{-1} \\
	\text{long-run restrictions on }\Theta(1)\\
	\end{bmatrix}$$
	is minimized.
	Normalize the shocks such that the diagonal elements of $B_0^{-1}$ are positive.
	\item Use the implied estimates of $B_0^{-1}$ and $\Sigma_\varepsilon$ to plot the structural impulse responses functions
	for (i) real GNP, (ii) the yield on Treasury Bills, (iii) the real interest rate and (iv) real money growth.
    Add 68\% and 95\% confidence intervals using a bootstrap approach.
\end{enumerate}

\paragraph{Readings}
\begin{itemize}
	\item \textcite{Gali_1992_HowWellDoes}
\end{itemize}

\begin{solution}\textbf{Solution to \nameref{ex:svarISLM}}
\ifDisplaySolutions
\input{exercises/svar_IS_LM.tex}
\fi
\newpage
\end{solution}
\fi
\newpage
\end{solution}
\fi
\newpage
\end{solution}
\fi
\newpage
\end{solution}