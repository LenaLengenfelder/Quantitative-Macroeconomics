\documentclass[10pt]{beamer}

\usetheme{metropolis} % Use the metropolis theme
\usepackage[utf8]{inputenc}
\usepackage[T1]{fontenc}

\title{Quantitative Macroeconomics}
\subtitle{Introduction}
\author{Willi Mutschler}
\institute{University of Tübingen}
\date{Winter Term 2023/2024}

\begin{document}

\maketitle


\begin{frame}
\frametitle{Course Overview}
\begin{itemize}
  \item Theoretical macroeconomics
  \begin{itemize}
    \item Understand dynamics of aggregated variables (economic growth, unemployment, inflation) by means of structural models
  \end{itemize}
  \item Econometric methods
  \begin{itemize}
    \item Apply statistical methods on macroeconomic time series data
  \end{itemize}
  \item Goals
  \begin{itemize}
    \item Theoretical and \textbf{methodological} foundations of state-of-the-art quantitative macroeconomics
    \item Enable understanding of macroeconomic studies
    \item Enable derivation and evaluation of own structural models
  \end{itemize}
  \item Focus on \textbf{quantitative/empirical} aspects of
  \begin{itemize}
    \item Structural Vector Autoregressive (SVAR) models
    \item Dynamic Stochastic General Equilibrium (DSGE) models
  \end{itemize}
  
\end{itemize}
\end{frame}

\begin{frame}
\frametitle{Topics I}
\textbf{Time Series Analysis}
\begin{enumerate}
    \item Fundamentals of macroeconomic time series data
    \item Fundamentals of univariate time series analysis (estimation, forecasting, evaluation)
    \item Fundamentals of multivariate time series data (VAR models)
\end{enumerate}
\end{frame}

\begin{frame}
\frametitle{Topics II}
\textbf{Structural Vector Autoregressive (SVAR) Models}
\begin{enumerate}
    \item Identification problem
    \item Identification By Exclusion Restrictions: recursive, short-run, long-run
    \item Asymptotic and bootstrap inference
    \item Introduction to Bayesian estimation (Gibbs sampler, Minnesota prior)
    \item Narrative identification in SVAR models
    \item Local Projections
\end{enumerate}
\end{frame}

\begin{frame}
\frametitle{Topics III}
\textbf{Dynamic Stochastic General Equilibrium (DSGE) Models}
\begin{enumerate}
    \item Algebra of RBC and New Keynesian Models
    \item First-order perturbation of DSGE models
    \item Kalman filter and smoother
    \item Estimation of linearized DSGE models (GMM, SMM, Maximum Likelihood, Bayesian MCMC)
\end{enumerate}
\end{frame}

\begin{frame}
\frametitle{Course Approach}
\begin{itemize}
  \item Highly computational and interactive course design  
  \item ``Flipped classroom light''
  \begin{itemize}
    \item No formal separation between lectures and exercises
    \item Weekly To-Do lists with reading, programming or video assignments
    \item Replication of case studies and papers
    \item Mostly we'll meet for Q\&A sessions and discussions, but also for theoretical inputs (see schedule on Ilias)
    \item Flexible office hours for individual questions via Zoom
  \end{itemize}
  \item Midterm and endterm exams (take-home, 2 weeks)
\end{itemize}
\end{frame}

\begin{frame}
\frametitle{Course Materials}
\begin{itemize}
  \item To-Do lists, exercises, and exams are available on \href{https://github.com/wmutschl/Quantitative-Macroeconomics}{GitHub} (PDFs are under \texttt{Releases})
  \item Solutions will be posted with a slight delay
  \item Literature (password protected) is available on Ilias
  \item Course communication via email, need to be subscribed to the course on Ilias
  \item All sessions are streamed live using Zoom; recordings are made available on Ilias
\end{itemize}
\end{frame}

\begin{frame}
\frametitle{Requirements}
\begin{itemize}
  \item Basic undergraduate knowledge of macroeconomics and econometrics
  \item Programming skills in a scripting language are advantageous, but not necessary (learning by doing)
\end{itemize}
\end{frame}

\begin{frame}
\frametitle{Software Used}
\begin{itemize}
  \item MATLAB or Octave
  \item Dynare
  \item Contributions in other programming languages are appreciated\\ \(\hookrightarrow\) extra credit (one grade step) if you submit at least six solutions
\end{itemize}
\end{frame}

\end{document}
