% !TeX encoding = UTF-8
% !TeX spellcheck = en_US
% !TEX root = endterm_exam_WS2324.tex
\documentclass{article}
%%%%%%%%%%%%%%%%%%%%%%%%%%%%%%%%%%%%%%%%%%%%%%%%%%%%%%%%%%%%%%%%%%%%%%%%%%%%%%%%%%%%%%%%%%%%%%%%%%%%%%%%%%%%%%%%%%%%%%%%%%%%%%%%%%%%%%%%%%%%%%%%%%%%%%%%%%%%%%%%%%%%%%%%%%%%%%%%%%%%%%%%%%%%%%%%%%%%%%%%%%%%%%%%%%%%%%%%%%%%%%%%%%%%%%%%%%%%%%%%%%%%%%%%%%%%
\usepackage[a4paper,top=2cm]{geometry}
\usepackage{amssymb,amsmath,amsfonts}
\usepackage[english]{babel}
\usepackage[a4paper]{geometry}
\usepackage{enumitem}
\usepackage{booktabs}
\usepackage{csquotes}
\usepackage{graphicx}
\usepackage[numbered,framed]{matlab-prettifier}
\usepackage{url}
\usepackage[backend=biber,style=authoryear]{biblatex}
\addbibresource{../literature/_biblio.bib}
\begin{document}
	
\title{Quantitative Macroeconomics}
\author{Endterm Exam}
\date{Winter 2023/2024}
\maketitle

\section*{General information}
\begin{itemize}
	\item Answer \textbf{three} out of the following \textbf{four} exercises in English.
	\item All assignments will be given the same weight in the final grade.
	\item Hand in your solutions before March 1, 2024 at 3pm.
	\item The solution files should contain your executable (and commented) script files
		as well as all additional documentation as \texttt{pdf}, not \texttt{txt}, \texttt{md}, \texttt{tex}, \texttt{doc} or \texttt{docx}.
	Your \texttt{pdf} files may also include scans or pictures of handwritten notes.
	\item Please e-mail ALL the solution files to \url{willi.mutschler@uni-tuebingen.de}.
	I will confirm the receipt of your work also by email (typically within the hour). If not, please resend it to me.
	\item All students must work on their own, please also give your student ID number and the name of the module you want to earn credits for.
	\item It is advised to regularly check Ilias and your emails in case of urgent updates.
	\item If there are any questions, do not hesitate to contact Willi Mutschler.
\end{itemize}

\newpage

\section{RBC model with leisure and non-Ricardian agents}
Consider the basic Real Business Cycle (RBC) model with leisure and non-Ricardian agents. Assume that there is a continuum of consumers given on the interval $[0,1]$. A proportion of the population, $\omega$, are Ricardian agents who have access to financial markets and are indexed by $i \in [0,\omega)$. The other part of the population, $1-\omega$, is composed of non-Ricardian agents who do not have access to financial markets and are indexed by $j \in (\omega,1]$.

A Ricardian household maximizes present as well as expected future utility
\begin{align*}
\underset{\{C_{i,t},I_{i,t},L_{i,t},K_{i,t}\}}{\max} E_t \sum_{s=0}^{\infty} \beta^{s} U_{i,t+s}
\end{align*}
with $\beta <1$ denoting the discount factor and $E_t$ is expectation given information at time $t$. The contemporaneous utility function 
\begin{align*}
U_{i,t} = \gamma \ln C_{i,t} + (1-\gamma) \ln{(1-L_{i,t})}
\end{align*}
has two arguments: consumption $C_{i,t}$ and labor $L_{i,t}$. The marginal utility of consumption is positive, whereas more labor reduces utility. Accordingly, $\gamma$ is the elasticity of substitution between consumption and labor. In each period the household takes the real wage $W_t$ as given and supplies perfectly elastic labor service to the representative firm. In return, she receives real labor income in the amount of $W_t L_{i,t}$ and, additionally, profits $\Pi_{i,t}$ from the firm as well as revenue from lending capital in the previous period $K_{i,t-1}$ at interest rate $R_t$ to the firms, as it is assumed that the firm and capital stock are owned by the Ricardian households. Income and wealth are used to finance consumption $C_{i,t}$ and investment $I_{i,t}$. In total, this defines the (real) budget constraint of the Ricardian agent:
\begin{align*}
C_{i,t} + I_{i,t} = W_t L_{i,t} + R_t K_{i,t-1} + \Pi_{i,t}
\end{align*}

The law of motion for capital $K_{i,t}$ at the end of period $t$ is given by
\begin{align*}
K_{i,t} = (1-\delta)K_{i,t-1} + I_{i,t}
\end{align*}
where $\delta$ is the depreciation rate. Assume that the transversality condition is full-filled.

A non-Ricardian household maximizes present as well as expected future utility
\begin{align*}
\underset{\{C_{j,t},L_{j,t}\}}{\max} E_t \sum_{s=0}^{\infty} \beta^{s} U_{j,t+s}
\end{align*}
The contemporaneous utility function is the same as for non-Ricardian households, i.e.
\begin{align*}
U_{j,t} = \gamma \ln C_{j,t} + (1-\gamma) \ln{(1-L_{j,t})}
\end{align*}
As non-Ricardian agents do not have access to the credit market, their (real) budget constraint is given by:
\begin{align*}
C_{j,t} = W_t L_{j,t}
\end{align*}

It is assumed that all agents, independently the group they belong to, are identical. Therefore, aggregate values (in per capita terms) are given by:
\begin{align*}
	C_t = \omega C_{i,t} + (1-\omega)C_{j,t}, && 		L_t = \omega L_{i,t} + (1-\omega)L_{j,t}, && K_t = \omega K_{i,t} &&	I_t = \omega I_{i,t}
\end{align*}
where the right two expressions are due to the fact that only Ricardian agents invest in physical capital.

Productivity $A_t$ is the driving force of the economy and evolves according to
\begin{align*}
\ln{A_{t}} &= \rho_A \ln{A_{t-1}}  + \varepsilon_t^A
\end{align*}
where $\rho_A$ denotes the persistence parameter and $\varepsilon_t^A$ is assumed to be normally distributed with mean zero and variance $\sigma_A^2$.

Real profits $\Pi_t = \omega \Pi_{i,t}$ of the representative firm are revenues from selling output $Y_t$ minus costs from labor $W_t L_t$ and renting capital $R_t K_{t-1}$:
\begin{align*}
\Pi_t = Y_{t} - W_{t} L_{t} - R_{t} K_{t-1}
\end{align*}	
The representative firm maximizes expected profits
\begin{align*}
\underset{\{L_{t},K_{t-1}\}}{\max} E_t \sum_{j=0}^{\infty} \beta^j Q_{t+j}\Pi_{t+j}
\end{align*}
subject to a Cobb-Douglas production function
\begin{align*}
Y_t = A_t K_{t-1}^\alpha L_t^{1-\alpha}
\end{align*}
The discount factor takes into account that firms are owned by the Ricardian households, i.e. $\beta^s Q_{t+s}$ is the present value of a unit of consumption in period $t+s$ or, respectively, the marginal utility of an additional unit of profit; therefore $Q_{t+s}=\frac{\partial U_{i,t+s}/\partial C_{i,t+s}}{\partial U_{i,t}/\partial C_{i,t}}$.

Finally, we have the non-negativity constraints	$K_{i,t} \geq0$, $C_{i,t} \geq 0$, $C_{j,t} \geq 0$, $0\leq L_{i,t} \leq 1$ and $0\leq L_{j,t} \leq 1$. Furthermore, clearing of the labor as well as goods market in equilibrium implies
\begin{align*}
Y_t = C_t + I_t
\end{align*}

\begin{enumerate}
	\item Briefly provide intuition behind the introduction of non-Ricardian households.
	\item Show that the first-order conditions of the agents are given by
		\begin{align*}
		E_t\left[\frac{C_{i,t+1}}{C_{i,t}}\right] = \beta E_t\left[1-\delta + R_{t+1}\right],&&		
		W_t = \frac{1-\gamma}{\gamma} \frac{C_{i,t}}{1-L_{i,t}},\\
		C_{j,t} = W_t L_{j,t}, &&
		W_t = \frac{1-\gamma}{\gamma} \frac{C_{j,t}}{1-L_{j,t}}.
		\end{align*}
	Interpret these equations in economic terms.
	
	\item Show that the first-order conditions of the representative firm are given by
	\begin{align*}
	W_t = (1-\alpha) A_t \left(\frac{K_{t-1}}{L_t}\right)^\alpha, &&	R_t = \alpha A_t \left(\frac{L_t}{K_{t-1}}\right)^{1-\alpha}
	\end{align*}
	Interpret these equations in economic terms.
	\item Discuss how to calibrate the parameters of the model.
	\item Write a DYNARE mod file for this model with a feasible calibration and compute the steady-state of the model either analytically or numerically.
	\item Compare the steady-state values of an economy with a high proportion of Ricardian agents versus one with a very low proportion.
	Try to interpret your results economically.
\end{enumerate} 

\newpage

\section[Maximum likelihood estimation of Ireland (2004)]{Maximum likelihood estimation of Ireland (2004)\label{ex:Ireland2004ML}}
\paragraph{Description}
Consider a version of the model of \textcite{Ireland_2004_TechnologyShocksNew} who assesses which shocks are the major drivers of aggregate fluctuations.
To this end, he augments a basic small-scale New-Keynesian model with preference, cost-push, monetary and technology shocks.

\paragraph{Data}
For the estimation, we will use quarterly U.S. data from 1948:Q1 to 2003:Q1. Specifically:
\begin{itemize}
	\item Quarterly changes in seasonally adjusted real GDP figures, converted to per capita values by dividing by the civilian noninstitutional population aged 16 and over, are used to measure \textbf{output growth}.
	\item Quarterly changes in the seasonally adjusted GDP deflator provide the measure of \textbf{inflation}.
	\item Quarterly averages of daily values of the three-month U.S. Treasury bill rate provide the measure of the \textbf{nominal interest rate}.
\end{itemize}

\paragraph{Parameters}
Of the 14 underlying parameters in the model, two are held fix, $\beta=0.99$ and $\psi=0.1$, and are not estimated.
Hence, our interest centers around the other 12 parameters which we will estimate by maximizing the log-likelihood function.

\paragraph{Codes}
A Dynare mod file is given in the file \texttt{ireland2004.mod} and \texttt{ireland2004data.mat} contains the dataset.

\paragraph{Exercises}
\begin{enumerate}
	\item What is the \emph{dilemma of absurd parameters} when doing a Maximum Likelihood estimation of DSGE models?
	\item Fix $\alpha_\pi$ to a very small number, e.g. 0.0001.
	Add an \texttt{estimated\_params} block with feasible starting values, lower and upper bounds for the remaining 11 parameters.
	Estimate the parameters with maximum likelihood.
	Comment on the estimation results in terms of point estimates and standard errors.
	\item Now also include $\alpha_\pi$ to the set of estimated parameters.
	How does this affect the estimation results in terms of point estimates and standard errors?
	Try to explain this and point towards possible solutions.
\end{enumerate}

\printbibliography

\newpage

\section{Kalman filter}
The first-order perturbation solution of a DSGE model is given by the following state-space representation:
\begin{align*}
\hat{y}_t &= g_y \hat{y}_{t-1} + g_u u_t, && u_t \sim \mathcal{N}(0,\Sigma_u)
\\
d_t &= \bar{d} + H \hat{y_t} + e_t, && e_t \sim \mathcal{N}(0,\Sigma_e)
\end{align*}
where \(\hat{y}_t = y_t - \bar{y}\).

\begin{enumerate}
	\item Write a MATLAB function \texttt{logLik = dsgeKalmanFilter(d,dSS,H,gy,gu,SIGu,SIGe)}
	that computes the log-likelihood for any DSGE model given a data matrix \(d\),
	the steady-state of observables \(dSS\), the matrices \(H\), \(g_y\), \(g_u\), and the covariance matrices \(\Sigma_u\) and \(\Sigma_e\).
	\item Load the data from the file \texttt{dsgeKalmanMatrices.mat} which contains the matrices \(g_y\), \(g_u\), \(H\), \(\Sigma_u\), \(\Sigma_e\) and \(\bar{d}\) as well as the data \(d\).
	Test your function with these inputs.
\end{enumerate}
\paragraph{Hints:}
In the lecture we have derived the necessary steps to implement the Kalman filter.
\begin{itemize}
\item Initialize $\hat{y}_{0|-1}=0$ and $\Sigma_{0|-1}$ as the solution to the Lyapunov equation:
$$\Sigma_{0|-1} = g_y \Sigma_{0|-1} g_y' + g_u \Sigma_u g_u'$$
Hint: You might want to use the auxiliary function \texttt{dlyapdoubling(gy,gu*SIGu*gu')} to solve this equation.
\item For $t=1,...,T$:
\begin{itemize}
	\item Compute the Kalman Gain using: $$K_t = g_y \Sigma_{t|t-1} H' \Omega_t^{-1}$$ where $\Omega=H \Sigma_{t|t-1} H' + \Sigma_e$		
	\item Compute the forecast error in the observations using: $$a_t=d_t - \bar{d} - H \hat{y}_{t|t-1}$$		
	\item Compute the state forecast for next period given today's information: $$\hat{y}_{t+1|t} = g_y \hat{y}_{t|t-1} + K_t a_t$$
	\item Update the covariance matrix: $$\Sigma_{t+1|t} = (g_y - K_t H) \Sigma_{t|t-1} (g_y - K_t H)' + g_u \Sigma_u g_u' + K_t \Sigma_e K_t'$$
	\item The time \(t\) contribution to the log-likelihood is:
	$$\log(f(d_t|d^{t-1})) = \frac{-n_d}{2}\log(2\pi) -\frac{1}{2}\log(\det(\Omega_t)) - \frac{1}{2} a_t' \Omega_t^{-1} a_t$$
\end{itemize}
\item The log-likelihood \texttt{logLik} is the sum of the individual contributions.
\end{itemize}



\newpage


\section{Random-Walk-Metropolis-Hastings algorithm for AR(1)}
Re-consider the univariate first-order autoregressive process:
\begin{align*}
	y_t = \rho y_{t-1} + u_t
\end{align*}
where we restrict \(0\leq \rho <1\) and \(u_t \sim \mathcal{N}(0,\sigma^2)\).
Simulated data for this process is given in the file \texttt{AR1Simulated.csv}.
\begin{enumerate}
	\item Briefly outline the Random-Walk-Metropolis-Hastings algorithm.
	What is the intuition behind this algorithm?
	\item What are common choices to initialize the proposal distribution of the Random-Walk-Metropolis-Hastings algorithm?
	Which one is preferred and why?
	\item Our goal is to estimate the parameters \(\rho \) and \(\sigma \) using the Random-Walk-Metropolis-Hastings algorithm.
	To this end:
	\begin{itemize}
		\item Assume that the prior distribution for
		\begin{itemize}
			\item \(\rho \) is Beta with both shape parameters being equal to 4: \texttt{betapdf(x,4,4)}.
			\item \(\sigma \) is Uniform on the interval \([0,2]\): \texttt{unifpdf(x,0,2)}.
		\end{itemize}
		Plot the prior distribution for both parameters and comment whether this is a good choice.
		\item Write a function \texttt{logposterior} that computes the log-posterior of the AR(1) process for any \(x=(\rho,\sigma)\) given the data.
		Hints:
		\begin{itemize}
			\item We have already derived the log-likelihood function in the lecture:\\\texttt{logLikeARpNorm(x,data,1,0)}.
			\item Don't forget the log on the prior distributions.
			\item As a cross-check \texttt{logposterior([0.5,0.3])} \(\approx 42.1050\).
		\end{itemize}		
		\item Initialize the proposal distribution at the posterior mode and use the inverse hessian for the jumping covariance matrix.
		Scale the jumping covariance by the factor \(c=2.5\).
		Hints:
		\begin{itemize}
			\item You need to run a numerical optimizer on \texttt{-1*logposterior} to find the mode.
			\item The inverse hessian is given by \texttt{inv(hessian)} where \texttt{hessian} is either the output of the optimizer
			or you can use the auxiliary function \texttt{hessian\_numerical} to compute the hessian numerically.
		\end{itemize}
		If you fail this step, use something else as the jumping covariance matrix.
		\item Initialize the chain \texttt{xpost} at the posterior mode and draw 100000 draws using the RW-MH algorithm.
		\begin{itemize}
			\item A new draw from the proposal is given by: \texttt{xtry = mvnrnd(xpost(:,i-1),c*SigmaMH)}
			\item Check whether \(\rho=xtry(1)\) and \(\sigma=xtry(2)\) are within the parameter space; if not penalize the log-posterior by -Inf.
			\item Do the Metropolis-Hastings accept-reject mechanism and store the accepted draws in \texttt{xpost}.
		\end{itemize}
		\item Remove the first 50\% of the draws.
		Plot the kernel density estimates of the marginal posterior distributions for \(\rho \) and \(\sigma \), e.g.\ \texttt{[frho, xrho] = ksdensity(rhopost);},
		and compare it to their prior distributions.
	\end{itemize}
\end{enumerate}

\newpage


\newpage

\appendix

\end{document}
